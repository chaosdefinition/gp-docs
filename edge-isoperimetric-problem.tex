\documentclass{article}

\usepackage[UTF8]{ctex}	% Chinese environment
\usepackage{amssymb}	% mathematical symbols

\author{L. H. Harper}
\date{2016 年 3 月 11 日}

\begin{document}

\title{边等周问题}
\maketitle

\section{基本定义}
\label{Section 1}

一个图 (Graph) $G = (V, E, \partial)$ 由顶点集 $V$、边集 $E$ 和标识每一条边对应一对顶点
(可以相同) 的边界函数 $\partial \colon E \rightarrow (_1^v) \cup (_2^v)$ 构成。
图经常用图表 (Diagram) 来表示,其中顶点对应图表中的点,边对应图表中连接一对点的曲线。
对任意的图 $G$ 和 $S \subseteq V$,我们定义:
$$\Theta(S) = \{e \in E \mid \partial(e) = \{v, w\}, v \in S, w \notin S\},$$
并称之为 $S$ 的边界 (Edge-boundary)。那么对于一个给定的图 $G$ 和 $k \in \mathbb{Z}^+$,
边等周问题 (Edge-Isoperimetric Problem, EIP) 是对于所有的 $S \in V$ 并且 $|S| = k$,
找出最小的 $|\Theta(S)|$。注意 $|\Theta(S)|$ 是不变量 (Invariant),即如果
$\phi \colon G \rightarrow H$ 表示一个图同构,那么对于 $\forall S \in V_G$,都有
$|\Theta(\phi(S))| = |\Theta(S)|$。因此,在一个自同构下等价的顶点子集有相同的边界。

自环,即只对应一个顶点的边,与 EIP 无关,因此我们将忽略它们。我们大多数的图 (但并非全部)
是简单图 (Ordinary Graph),即不含自环和多重边。一个简单图的表示可以缩短至 $(V, E)$,
其中 $E \in (_2^v)$,$\partial$ 则是隐含的。

\section{例子}
\label{Section 2}

\subsection{$K_n$,$n$-完全图}
\label{Subsection 2.1}

$K_n$ 有 $n$ 个顶点且 $E = (_2^v)$,即每一对不同顶点之间都有一条边。对于每一个 $S \in V$
且 $|S| = k$,$|\Theta(S)| = |S \times (V − S)| = k(n − k)$。因此 $K_n$ 的 EIP
很简单:任意一个 $k$-集都是解。

\subsection{$\mathbb{Z}_n$,$n$-环}
\label{Subsection 2.2}

对于 $\mathbb{Z}_n$,有 $V = \{0, 1, \dots, n − 1\}$ 和
$E = \{\{i, j\} \mid i − j \equiv  \pm 1(\models n)\}$。因此,$Z_3 = K_3$,$Z_4$
的图表见。

\end{document}
