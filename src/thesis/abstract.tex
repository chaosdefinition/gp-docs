%% LaTeX source of Abstract of the thesis.
%% NEVER compile this file. Complie 'thesis.tex' instead.

\chapter*{摘要}
\label{Abstract CN}

从超立方体到环的嵌入在很长一段时间里都是一个开问题。
Ching-Jung Guu 在她的博士学位论文 \cite{Guu.1997} 中宣称从超立方体到环嵌入的线长问题已经得到了解决,
她通过创建派生网络这一概念将该问题转化为了边等周问题,
而后者则已经在 L. H. Harper 的论文 \cite{Harper.1964} 中得到解决。
Guu 依据此得出了格雷码嵌入有最小的环形线长这一结论。
然而,Jason Erbele 等人 在 \cite{Erbele.2003} 中指出了她的证明过程中的错误,
这导致 Guu 的结论可能不成立。

在本文中,我们开发了一个分布式计算程序——我们称之为 Hyperspark——用来探索较低维度下
从超立方体到环的嵌入的所有可能性。
我们将用我们程序的运行结果验证 Guu 的结论的正确性,
并且试图寻找最优嵌入的一些新特性。
这些新特性或许能够为该问题提供一些线索,甚至能够启发整个问题的解决。
\hfill\break

\textbf{关键词:} 超立方体;环嵌入;线长问题;分布式计算

\chapter*{Abstract}
\label{Abstract EN}

Embedding of hypercubes into circles has been an open problem for a long time.
It was claimed by Ching-Jung Guu in her Ph.D. dissertation \cite{Guu.1997} that
the circular wirelength problem of hypercubes was solved by creating a Derived
Network that translates the problem into an Edge Isoperimetric Problem,
which was solved by L. H. Harper in \cite{Harper.1964}, based on which
she drew a conclusion that the Graycode embedding minimizes the circular wirelength.
However, errors in her proof have been spotted by Jason Erbele et al. in \cite{Erbele.2003},
which may make Guu's conclusion untenable.

In this thesis, we develop a distributed computing program called Hyperspark to
explore the possibilities of all embeddings of hypercubes into circles under lower dimensions.
We will verify the correctness of Guu's conclusion using the results of our program,
and try to find out some new traits of optimal embeddings,
which may provide clues to the problem or even inspire the entire solution.
\hfill\break

\textbf{Keywords:} Hypercube; Circular embedding; Wirelength problem;
Distributed computing
