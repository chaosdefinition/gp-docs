%% LaTeX source of Appendices of the thesis.
%% NEVER compile this file. Complie 'thesis.tex' instead.

\appendix{Hyperspark 核心源代码}
\label{Appendix A}

\section{App.java}
\label{Section A.1}

\begin{lstlisting}[language = Java]
package me.chaosdefinition.hyperspark;

import java.util.List;

import org.apache.spark.SparkConf;
import org.apache.spark.api.java.JavaRDD;
import org.apache.spark.api.java.JavaSparkContext;

import me.chaosdefinition.hyperspark.common.HypersparkException;
import me.chaosdefinition.hyperspark.core.WirelengthCalculator;
import me.chaosdefinition.hyperspark.optparse.OptionParser;
import me.chaosdefinition.hyperspark.util.MathUtils;

/**
 * Application starting point.
 *
 * @author Chaos Shen
 */
public class App {
	public static void main(String[] args) {
		try {
			/* Parse command line options. */
			OptionParser parser = new OptionParser();
			parser.parse(args);

			/* Get options. */
			int dimension = parser.getDimension();
			int length = parser.getLength() < 3 ? 3 : parser.getLength();
			String outdir = parser.getOutdir();
			boolean verbose = parser.isVerbose();

			/* Initialize Spark environment. */
			SparkConf conf = new SparkConf().setAppName("hyperspark");
			JavaSparkContext ctx = new JavaSparkContext(conf);

			/*
			 * Generate a list of lexicographic ordered integers, each
			 * representing an index in all permutations of specified length
			 * without rotation and flip symmetries, and then parallelize them
			 * to JavaRDD.
			 */
			JavaRDD<Integer> indices = ctx.parallelize(
					MathUtils.lexicode(MathUtils.factorial(length - 1) / 2)
			);

			if (verbose) {
				/*
				 * In verbose mode, first convert each index to corresponding
				 * initial permutation, and then construct a
				 * WirelengthCalculator with specified dimension and the initial
				 * permutation, finally call findMinimumMappings() which returns
				 * the result by this permutation.
				 *
				 * We don't need to reduce all the results since they are
				 * exactly what we want. Save them to the output directory.
				 */
				indices.flatMap(index -> {
					List<Integer> initial = MathUtils.indexToPermutation(index,
						length
					);
					WirelengthCalculator calculator = new WirelengthCalculator(
						dimension, initial
					);
					calculator.setVerbose(verbose);
					return calculator.findMinimumMappings();
				}).saveAsTextFile(outdir);
			} else {
				/*
				 * Otherwise, we just need to know the minimum wirelength, which
				 * can be acquired by following the above steps but finally
				 * calling calculateMinimumWirelength() instead.
				 *
				 * Reduce the results by taking the minimum and print the final
				 * minimum.
				 */
				int minimum = indices.map(index -> {
					List<Integer> initial = MathUtils.indexToPermutation(
						index, length
					);
					WirelengthCalculator calculator = new WirelengthCalculator(
						dimension, initial
					);
					return calculator.calculateMinimumWirelength();
				}).reduce((v1, v2) -> Math.min(v1, v2));
				System.out.println("Minimum: " + minimum);
			}
			ctx.close();
		} catch (Exception e) {
			if (e instanceof HypersparkException) {
				e.printStackTrace();
			} else {
				new HypersparkException(e).printStackTrace();
			}
			System.exit(1);
		}
	}
}
\end{lstlisting}

\clearpage
\section{WirelengthCalculator.java}
\label{Section A.2}

\begin{lstlisting}[language = Java]
package me.chaosdefinition.hyperspark.core;

import java.util.ArrayList;
import java.util.Arrays;
import java.util.List;

import me.chaosdefinition.hyperspark.common.HypersparkException;

/**
 * The main class of hyperspark.
 *
 * @author Chaos Shen
 */
public class WirelengthCalculator {

	private int dimension;
	private int vertices;
	private List<Integer> initial;
	private int minWirelength;
	private List<List<Integer>> minMappings;

	private boolean verbose = false;

	/**
	 * Determines if two vertices have an edge between them.
	 *
	 * @param v1
	 *            the integer representation of a vertex
	 * @param v2
	 *            the same as {@code v1}
	 * @return {@code true} if the two vertices have an edge between them,
	 *         otherwise {@code false}
	 */
	public static boolean hasEdge(int v1, int v2) {
		/*
		 * An egde exists if only one bit differs, so we do this check by
		 * determing if their xor sum is a power of two.
		 */
		return (v1 ^ v2) != 0 && ((v1 ^ v2) & ((v1 ^ v2) - 1)) == 0;
	}

	/**
	 * Returns the known minimum wirelength of circular mappings of
	 * hypercube of specified dimension, which typically is the
	 * wirelength of the graycode mapping.
	 *
	 * @param dimension
	 *            the dimension of the hypercube
	 * @return the known minimum circular wirelength
	 */
	public static int knownMinimumWirelength(int dimension) {
		if (dimension < 0) {
			throw new HypersparkException(
				"Illegal value of dimension: " + dimension
			);
		} else if (dimension < 2) {
			return dimension;
		} else {
			return 3 * (1 << ((dimension << 1) - 3)) - (1 << (dimension - 1));
		}
	}

	/**
	 * Constructs a {@link WirelengthCalculator} for hypercube of specified
	 * dimension.
	 *
	 * @param dimension
	 *            the dimension of the hypercube
	 */
	public WirelengthCalculator(int dimension) {
		this(dimension, null);
	}

	/**
	 * Constructs a {@link WirelengthCalculator} for hypercube of specified
	 * dimension with a given initial list to start with.
	 *
	 * @param dimension
	 *            the dimension of the hypercube
	 * @param initial
	 *            the initial list to start with
	 */
	public WirelengthCalculator(int dimension, List<Integer> initial) {
		this.dimension = dimension;
		this.vertices = 1 << dimension;
		if (initial != null) {
			/* check duplication of each entry */
			boolean[] presence = new boolean[initial.size()];
			for (int i : initial) {
				if (i >= 0 && i < initial.size()) {
					presence[i] = true;
				}
			}
			for (boolean b : presence) {
				if (!b) {
					throw new HypersparkException(
						"Illegal value in initial list!"
					);
				}
			}
			this.initial = new ArrayList<>(initial);
		} else {
			/* [0, 1, 2] will produce all permutations without symmetry */
			this.initial = new ArrayList<>(Arrays.asList(0, 1, 2));
		}

		this.minWirelength = knownMinimumWirelength(dimension);
	}

	/**
	 * Calculates the minimum circular wirelength on all mappings under the
	 * initial list.
	 *
	 * @return the minimum circular wirelength
	 */
	public int calculateMinimumWirelength() {
		backtrack(this.initial, this.initial.size());
		return this.minWirelength;
	}

	/**
	 * Finds all the mappings of minimum circular wirelength under the
	 * initial list.
	 *
	 * @return a list of minimum mappings
	 */
	public List<List<Integer>> findMinimumMappings() {
		this.minMappings = new ArrayList<>();
		if (!this.verbose) {
			setVerbose(true);
			backtrack(this.initial, this.initial.size());
			setVerbose(false);
		} else {
			backtrack(this.initial, this.initial.size());
		}
		return this.minMappings;
	}

	/**
	 * Calculates the complete circular wirelength of a list.
	 *
	 * @param list
	 *            a list (an initial segment or a complete mapping of
	 *            hypercube)
	 * @return the complete circular wirelength
	 */
	public int calculateWirelength(List<Integer> list) {
		return calculateWirelength(list, 0, list.size());
	}

	/**
	 * Calculates the partial circular wirelength of a list.
	 *
	 * @param list
	 *            a list (an initial segment or a complete mapping of
	 *            hypercube)
	 * @param start
	 *            the starting position in the list
	 * @param end
	 *            the ending position in the list
	 * @return the partial circular wirelength, or -1 when partial
	 *         wirelength is already larger than the current minimum
	 */
	public int calculateWirelength(List<Integer> list, int start, int end) {
		int wirelength = 0;

		for (int i = start; i < end; ++i) {
			for (int j = i + 1; j < end; ++j) {
				if (hasEdge(list.get(i), list.get(j))) {
					if ((j - i) * 2 > this.vertices) {
						wirelength += this.vertices - (j - i);
					} else {
						wirelength += j - i;
					}
					if (wirelength > this.minWirelength) {
						return -1;
					}
				}
			}
		}

		return wirelength;
	}

	/**
	 * Performs backtracking to generate all the mappings under the
	 * initial list.
	 *
	 * @param list
	 *            a list (an initial segment or a complete mapping of
	 *            hypercube)
	 * @param element
	 *            next element to add to the list
	 */
	private void backtrack(List<Integer> list, int element) {
		if (element >= this.vertices) {
			this.minWirelength = calculateWirelength(list);
			if (this.verbose) {
				this.minMappings.add(new ArrayList<>(list));
			}
		} else {
			for (int i = 1; i <= list.size(); ++i) {
				list.add(i, element);
				if (calculateWirelength(list) != -1) {
					backtrack(list, element + 1);
				}
				list.remove(i);
			}
		}
	}

	/* getters and setters */
	public int getDimension() {
		return dimension;
	}

	public int getVertices() {
		return vertices;
	}

	public List<Integer> getInitial() {
		return initial;
	}

	public int getMinWirelength() {
		return minWirelength;
	}

	public boolean isVerbose() {
		return verbose;
	}

	public void setVerbose(boolean verbose) {
		this.verbose = verbose;
	}
}
\end{lstlisting}

\clearpage
\section{MathUtils.java}
\label{Section A.3}

\begin{lstlisting}[language = Java]
package me.chaosdefinition.hyperspark.util;

import java.util.ArrayList;
import java.util.Arrays;
import java.util.List;

import me.chaosdefinition.hyperspark.common.HypersparkException;

/**
 * Relevant mathematical utilities.
 *
 * @author Chaos Shen
 */
public class MathUtils {

	/**
	 * Generates the factorial of a given integer.
	 *
	 * @param n
	 *            an integer
	 * @return its factorial
	 */
	public static int factorial(int n) {
		final int[] factorials = {
			1, 1, 2, 6, 24, 120, 720, 5040, 40320, 362880, 3628800, 39916800,
			479001600
		};

		if (n >= factorials.length) {
			throw new HypersparkException("Factorial of " + n + "too large!");
		}

		return factorials[n];
	}

	/**
	 * Converts an integer to a permutation of given length without
	 * rotation and flip symmetries.
	 *
	 * @param index
	 *            an integer ranging from 0 to ((n - 1)! / 2 - 1)
	 * @param length
	 *            the length of the target permutation
	 * @return the target permutation
	 */
	public static List<Integer> indexToPermutation(int index, int length) {
		List<Integer> code = new ArrayList<>(Arrays.asList(0, 1, 2));
		int[] position = new int[length];

		for (int i = length - 1; i >= 3; --i) {
			/* 0 is always fixed, so we need to shift others each by 1 */
			position[i] = index % i + 1;
			index /= i;
		}
		for (int i = 3; i < length; ++i) {
			code.add(position[i], i);
		}

		return code;
	}

	/**
	 * Generates the lexicographic code of given length.
	 *
	 * @param length
	 *            the length of the code
	 * @return the lexicographic code
	 */
	public static List<Integer> lexicode(int length) {
		List<Integer> code = new ArrayList<>(length);
		for (int i = 0; i < length; ++i) {
			code.add(i);
		}
		return code;
	}
}
\end{lstlisting}

\appendix{Matlab 画图程序核心源代码}
\label{Appendix B}

\section{main.m}
\label{Section B.1}

\begin{lstlisting}[language = Matlab]
axis equal;
hold on;

% open data file
[file, path] = uigetfile('*');
if file == 0
	quit;
end
fileID = fopen(strcat(path, file));

while 1
	% read a line, each line representing a result
	line = fgetl(fileID);
	if line == -1
		break;
	end
	data = sscanf(line, '%*c%d')';
	total = size(data, 2);
	cla;

	% draw a basic inner circle
	innerTheta = 0 : pi / 10000 : 2 * pi;
	innerX = 1 * sin(innerTheta);
	innerY = 1 * cos(innerTheta);
	plot(innerX, innerY, 'k');

	% draw the marks
	markTheta = 0 : 2 * pi / total : 2 * pi;
	markX = [];
	markY = [];
	for j2 = 1 : total
		markX = [1, 0.9, 0.8] * sin(markTheta(j2));
		markY = [1, 0.9, 0.8] * cos(markTheta(j2));
		plot(markX([1, 2]), markY([1, 2]), 'k');
		text(markX(3), markY(3), num2str(data(j2)));
	end

	% draw the outside arcs
	counters = zeros(1, total);
	heights = zeros(1, total);
	colors = ['r', 'g', 'b', 'm', 'y', 'c'];
	for i = 0 : log2(total) - 1
		for j = 1 : total / 2
			for k = 1 : total
				if j == 8 && k > 7
					continue;
				end
				j2 = mod(k + j - 1, total) + 1;
				if hasEdge(data(k), data(j2)) && ...
						bitxor(data(k), data(j2)) == 2^i
					[counters, heights] = drawArc(counters, heights, ...
							k, j2, total, ...
							colors(i + 1));
				end
			end
		end
	end

	% listen to keyboard to choose the next action
	while 1
		try
			if waitforbuttonpress
				action = get(gcf, 'CurrentCharacter');
			end
		catch
			action = 'q';
		end
		switch action
		case 'n'
			break;
		case 'q'
			fclose(fileID);
			quit;
		end
	end
end

fclose(fileID);
quit;
\end{lstlisting}

\clearpage
\section{drawArc.m}
\label{Section B.2}

\begin{lstlisting}[language = Matlab]
%% drawArc: draw an arc
function [counters, heights] = drawArc(counters, heights, i1, i2, ...
		total, color)
	maximum = 0;

	if i1 > i2
		i2 = i2 + total;
	end
	for i = i1 : i2 - 1
		index = mod(i - 1, total) + 1;
		if heights(index) > maximum
			maximum = heights(index);
		end
	end
	maximum = maximum + 1;
	for i = i1 : i2 - 1
		index = mod(i - 1, total) + 1;
		heights(index) = maximum;
		counters(index) = counters(index) + 1;
	end

	radius = 1 + maximum * 0.05;
	delta = pi / 80;
	degree = 2 * pi / total;
	theta = (i1 - 1) * degree + delta : pi / 1000 : (i2 - 1) * degree - delta;
	arcX = radius * sin(theta);
	arcY = radius * cos(theta);
	plot(arcX, arcY, color);
\end{lstlisting}

\clearpage
\section{hasEdge.m}
\label{Section B.3}

\begin{lstlisting}[language = Matlab]
%% hasEdge: determine if two vertices have an edge between them
function [v] = hasEdge(a, b)
	xorSum = bitxor(a, b);
	v = xorSum ~= 0 & bitand(xorSum, xorSum - 1) == 0;
	return;
\end{lstlisting}
