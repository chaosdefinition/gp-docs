%% LaTeX source of Chapter 1 of the thesis.
%% NEVER compile this file. Complie 'thesis.tex' instead.

\chapter{引言}
\label{Chapter 1}

图嵌入问题在计算机科学中有着重要的应用。
一方面,现代高性能并行计算机中的任务映射可以被建模为图嵌入问题:
首先它将任务作为节点,将任务间的通讯作为边,就能得到一个任务图;
接下来它将映射模拟为从一个图到另一个图的嵌入,并试图寻找这个嵌入的最小线长。
另一方面,多处理器的排列和连线也能转化为图嵌入问题。
一般意义上,图嵌入问题都是 NP 完全的 \cite{Garey.1979},
我们只能利用启发式方法来探究一些特殊的图结构用以应用到这些研究中,
其中超立方体是一种重要且高效的并行结构,但是其缺点是连线复杂;
而链、环、圆柱和网格等结构都具有连线简单的特点。
尽管有很多常规的图嵌入问题已经得以解决了,
如从超立方体到链 \cite{Harper.1964}、从超立方体到网格 \cite{Manuel.2009}、
从二叉树到网格 \cite{Opatrny.2000} 等,
然而从超立方体到环的嵌入则依然是一个开问题,
从超立方体到圆柱的嵌入则依赖于到环的研究 \cite{Ji.2015}。

分布式计算则是计算机科学中的另一热门领域。
分布式计算技术为研发高性能计算机开辟了另一条路径:
普通计算机通过网络连接之后,
可以利用这种技术成为分布式计算集群中的一个节点,
用以完成先前只能在传统意义上的高性能计算机(如超级计算机)上进行的工作。
随着科学技术的发展和进步,
分布式计算已经使参与其中的所有普通计算机的“联合计算能力”超过了单台超级计算机。
分布式计算虽然已经有几十年的研究历史,但现在依然是计算机研究领域中的一片热土,
并且已被广泛地运用于各种需要高强度计算能力的应用中,
如科学计算、海量数据分析处理等。

本文将对从超立方体到环的嵌入性质进行研究。
首先,我们在第 \ref{Chapter 2} 章对本文所用到的术语和背景知识做简单的介绍。
接下来在第 \ref{Chapter 3} 章,
我们简要地概括和说明现有的从超立方体到链和环的嵌入研究,
并给出对其中一部分结果的改进。
第 \ref{Chapter 4}、\ref{Chapter 5} 章是本文的主要内容:
我们在第 \ref{Chapter 4} 章介绍分布式计算程序 Hyperspark,
利用其对 $3$ 阶和 $4$ 阶超立方体进行嵌入和线长的计算,
并在第 \ref{Chapter 5} 章对计算结果做简单的分析。
第 \ref{Chapter 6} 章则是对本文的总结,以及对未来在该研究领域进一步工作的展望。
