%% LaTeX source of Chapter 1 of the thesis.
%% NEVER compile this file. Complie 'thesis.tex' instead.

\chapter{引言}
\label{Chapter 1}

一方面,现代高性能并行计算机中的任务映射可以被建模为图嵌入问题:
首先它将任务作为节点,将任务间的通讯作为边,就能得到一个任务图;
接下来它将映射模拟为从一个图到另一个图的嵌入,并试图寻找这个嵌入的最小线长。
另一方面,多处理器排列和连线也能转化为图嵌入问题。
有很多有用的图结构可以应用到这些研究中,
其中超立方体是一种重要且高效的并行结构,但是其缺点是连线复杂;
链、环、圆柱和网格等结构都具有连线简单的特点。
尽管有很多常规的图嵌入问题已经得以解决了,
如从超立方体到链 \cite{Harper.1964}、从超立方体到网格 \cite{Manuel.2009}、
从二叉树到网格 \cite{Opatrny.2000} 等,
然而从超立方体到环的嵌入则依然是一个开问题,
从超立方体到圆柱的嵌入则依赖于到环的研究。

本文将对从超立方体到环的嵌入性质进行研究。
首先,我们在第 \ref{Chapter 2} 章对本文所用到的术语和背景知识做简单的介绍。
接下来在第 \ref{Chapter 3} 章,
我们简要地概括和说明现有的从超立方体到链和环的嵌入研究,
并给出对其中一部分结果的改进。
第 \ref{Chapter 4}、\ref{Chapter 5} 章是本文的主要内容:
我们在第 \ref{Chapter 4} 章介绍分布式计算程序 Hyperspark,
利用其对 $3$ 阶和 $4$ 阶超立方体进行嵌入和线长的计算,
并在第 \ref{Chapter 5} 章对计算结果做简单的分析。
第 \ref{Chapter 6} 章则是对本文的总结,以及对未来在该研究领域进一步工作的展望。
