%% LaTeX source of Chapter 2 of the thesis.
%% NEVER compile this file. Complie 'thesis.tex' instead.

\chapter{背景知识}
\label{Chapter 2}

\section{图}
\label{Section 2.1}

一个图(Graph)$G = (V_G, E_G, \partial_G)$ 是一个三元组,
其中 $V_G$ 是顶点的集合,$E_G$ 是边的集合,
$\partial_G \colon E_G \rightarrow \binom{V_G}{1} \cup \binom{V_G}{2}$
则是从边集到顶点集的可重复二元子集的一个映射函数。
$\partial_G$ 指出了一条边对应了哪两个顶点。

$d$ 阶超立方体(Hypercube),我们用 $Q_d$ 来表示。
$Q_d = (V_{Q_d}, E_{Q_d}, \partial_{Q_d})$,
其中 $V_{Q_d} = \{0, 1\}^d$
(即 $\{0, 1\}$ 的 $d$ 倍笛卡尔积,$|V_{Q_d}| = 2^d$)。
对于 $e \in E_{Q_d}$ 和 $v, w \in V_{Q_d}$,
$\partial_{Q_d}(e) = \{v, w\}$ 当且仅当 $v$ 和 $w$ 有正好一位坐标不相同。
$Q_3$ 的图表见图 \ref{Figure 1}。

\begin{figure}[h!]
	\centering
	\includesvg{figure-1}
	\caption{$Q_3$ 的图表}
	\label{Figure 1}
\end{figure}

长为 $n$ 的环(Circle),我们用 $C_n$ 来表示。
$C_n = (V_{C_n}, E_{C_n}, \partial_{C_n})$,
其中 $V_{C_n} = \{0, 1, 2, \dots, n - 1\}$。
对于 $e \in E_{C_n}$ 和 $i, j \in V_{C_n}$,
$\partial_{C_n}(e) = \{i, j\}$ 当且仅当 $j \equiv (i + 1) \bmod n$。
$C_4$ 的图表见图 \ref{Figure 2}。

\begin{figure}[h!]
	\centering
	\includesvg{figure-2}
	\caption{$C_4$ 的图表}
	\label{Figure 2}
\end{figure}

给定图 $G = (V_G, E_G, \partial_G)$ 和 $v, w \in V_G$,
一条从 $v$ 到 $w$ 且长度为 $n$ 的路径(Path)$P(v, w)$ 是一个长为 $n$ 的边序列
$(e_1, e_2, \dots, e_n)$,
其中存在顶点序列 $(v_0 = v, v_1, \dots, v_{n - 1}, v_n = w)$
满足对于 $i = 1, \dots, n$,$e_i = \{v_{i - 1}, v_i\}$。
因此 $P(v, w)$ 也可以表示为 $(v e_1 v_1 e_2 \cdots e_n w)$。
从 $v$ 到 $w$ 的距离(Distance)$|P_G(v, w)|$ 是从 $v$ 到 $w$ 的最短路径的长度。

\section{线长和割宽}
\label{Section 2.2}

有了上述关于图的基本概念的定义,我们便可以来定义线长和割宽等一系列概念了。

给定图 $G = (V_G, E_G, \partial_G)$、
图 $H = (V_H, E_H, \partial_H)$ 和单射函数 $\eta \colon V_G \rightarrow V_H$,
其中 $|V_G| \le |V_H|$,那么在 $\eta$ 下从 $G$ 到 $H$ 的线长(Wirelength)为
\begin{equation*}
wl(G, H, \eta) = \sum_{\substack{
	e \in E_G \\
	\partial_G(e) = \{v, w\}
}} |P_H(\eta(v), \eta(w))|
\end{equation*}
即 $G$ 中所有的边在 $\eta$ 下嵌入到 $H$ 后的距离之和。
从 $G$ 到 $H$ 的线长为
\begin{equation*}
wl(G, H) = \min_{\eta} wl(G, H, \eta)
\end{equation*}
从 $G$ 到 $H$ 的线长问题即寻找 $wl(G, H)$ 和能达到最小线长的单射函数 $\eta$。

与线长类似,给定图 $G$、$H$ 和单射函数 $\eta$,
在 $\eta$ 下从 $G$ 到 $H$ 的割宽(Cutwidth)为
\begin{equation*}
cw(G, H, \eta) = \max_{i \in V_H} |\{\{v, w\} \in E_G \colon \\
P_H(\eta(v), \eta(w)) = (\cdots i \cdots), i \neq \eta(w)\}|
\end{equation*}
即穿过 $H$ 中的顶点的 $G$ 中边数的最大值。
从 $G$ 到 $H$ 的割宽为
\begin{equation*}
cw(G, H) = \min_{\eta} cw(G, H, \eta)
\end{equation*}
从 $G$ 到 $H$ 的割宽问题即寻找 $cw(G, H)$ 和能达到最小割宽的单射函数 $\eta$。

本文所研究的问题为从超立方体到环嵌入的线长问题,
即对 $\forall d > 2$,寻找 $wl(Q_d, C_{2^d})$ 和能达到最小线长的双射函数 $\eta$。
注意 $\eta$ 在这里也被称为编号方式(Numbering),
因为环的顶点集由自然数构成。

\section{格雷码编号方式}
\label{Section 2.3}

格雷码(Graycode)在计算机组成原理、体系结构以及通信领域中有着重要的作用,
并且研究人员已证明(或相信)
格雷码编号方式是从超立方体到链、环、圆柱等结构的嵌入的线长问题的解
\cite{Harper.1964,Guu.1997,Manuel.2011}。

对于从超立方体到环的嵌入,
格雷码编号方式 $\mathcal{G} \colon Q_d \rightarrow C_{2^d}$ 有如下的递归定义:
\begin{enumerate}[(1)]
	\item 首先我们有 $Q_1$ 上的格雷码编号方式
		$\mathcal{G} \colon Q_1 \rightarrow C_2$ 为
		\begin{equation*}
		\mathcal{G}(0) = 0 \qquad \mathcal{G}(1) = 1
		\end{equation*}
	\item 如果给定了 $Q_d$ 上的格雷码编号方式
		$\mathcal{G} \colon Q_d \rightarrow C_{2^d}$($d \ge 1$),
		那么我们给 $Q_d$ 的每一个顶点添加一个坐标前缀 $0$ 和一个相邻的顶点。
		这些新的顶点则给出了一个 $Q_d$ 的拷贝,
		我们为其添加一个坐标前缀 $1$。
		这个新的 $Q_d$ 的编号与原先的相反,
		即与编号为 $2^d - 1$ 的顶点相邻的顶点编号为 $2^d$、
		与编号为 $2^d - 2$ 的顶点相邻的顶点编号为 $2^d + 1$、……
		与编号为 $0$ 的顶点相邻的顶点编号为 $2^{d + 1} - 1$。
		这样我们便得到了 $Q_{d + 1}$ 上的格雷码编号方式了。
\end{enumerate}

