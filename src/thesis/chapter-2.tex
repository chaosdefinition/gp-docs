%% LaTeX source of Chapter 2 of the thesis.
%% NEVER compile this file. Complie 'thesis.tex' instead.

\chapter{背景知识}
\label{Chapter 2}

\section{图}
\label{Section 2.1}

一个\emph{图}(Graph)$G = (V_G, E_G, \partial_G)$ 是一个三元组,
其中 $V_G$ 是顶点的集合,$E_G$ 是边的集合,
$\partial_G \colon E_G \rightarrow \binom{V_G}{1} \cup \binom{V_G}{2}$
则是从边集到顶点集的可重复二元子集的一个映射函数。
$\partial_G$ 指出了一条边对应了哪两个顶点。
下面我们列出了三种在本文中会用到的图结构。

$d$ 阶\emph{超立方体}(Hypercube),我们用 $Q_d$ 来表示。
$Q_d = (V_{Q_d}, E_{Q_d}, \partial_{Q_d})$,
其中 $V_{Q_d} = \{0, 1\}^d$
(即 $\{0, 1\}$ 的 $d$ 倍笛卡尔积,$|V_{Q_d}| = 2^d$)。
对于 $e \in E_{Q_d}$ 和 $v, w \in V_{Q_d}$,
$\partial_{Q_d}(e) = \{v, w\}$ 当且仅当 $v$ 和 $w$ 有正好一位坐标不相同。
$Q_3$ 的示意图见图 \ref{Figure 2-1}。

\begin{figure}[h!]
	\centering
	\includesvg{figure-2-1}
	\caption{$Q_3$ 的示意图}
	\label{Figure 2-1}
\end{figure}

长为 $n$ 的\emph{链},也称为\emph{路径}(Path),我们用 $P_n$ 来表示。
$P_n = (V_{P_n}, E_{P_n}, \partial_{P_n})$,
其中 $V_{P_n} = \{0, 1, 2, \dots, n - 1\}$。
对于 $e \in E_{P_n}$ 和 $i, j \in V_{P_n}$,
$\partial_{C_n}(e) = \{i, j\}$ 当且仅当 $j = i + 1$。
$P_4$ 的示意图见图 \ref{Figure 2-2}。

\begin{figure}[h!]
	\centering
	\includesvg{figure-2-2}
	\caption{$P_4$ 的示意图}
	\label{Figure 2-2}
\end{figure}

长为 $n$ 的\emph{环}(Circle),我们用 $C_n$ 来表示。
$C_n = (V_{C_n}, E_{C_n}, \partial_{C_n})$,其中 $V_{C_n} = V_{P_n}$。
对于 $e \in E_{C_n}$ 和 $i, j \in V_{C_n}$,
$\partial_{C_n}(e) = \{i, j\}$ 当且仅当 $j \equiv (i + 1) \bmod n$。
$C_4$ 的示意图见图 \ref{Figure 2-3}。

\begin{figure}[h!]
	\centering
	\includesvg{figure-2-3}
	\caption{$C_4$ 的示意图}
	\label{Figure 2-3}
\end{figure}

给定图 $G = (V_G, E_G, \partial_G)$ 和 $v, w \in V_G$,
一条从 $v$ 到 $w$ 且长度为 $n$ 的路径 $P(v, w)$ 是一个导出自 $G$ 的长为 $n$ 的链,
其中 $v$ 可视为 $0$,$w$ 可视为 $n - 1$,$|P(v, w)| = n$。
从 $v$ 到 $w$ 的\emph{距离}(Distance)$d_G(v, w)$ 是从 $v$ 到 $w$ 的最短路径的长度。

\section{线长问题}
\label{Section 2.2}

有了上述关于图的基本概念的定义,我们便可以来定义线长等一系列概念了。

给定图 $G = (V_G, E_G, \partial_G)$、
图 $H = (V_H, E_H, \partial_H)$ 和单射函数 $\eta \colon V_G \rightarrow V_H$,
其中 $|V_G| \le |V_H|$,那么在 $\eta$ 下从 $G$ 到 $H$ 的\emph{线长}(Wirelength)为
\begin{equation}
wl(G, H, \eta) = \sum_{\substack{
	e \in E_G \\
	\partial_G(e) = \{v, w\}
}} d_H(\eta(v), \eta(w))
\end{equation}
即 $G$ 中所有的边在 $\eta$ 下嵌入到 $H$ 后的距离之和。
从 $G$ 到 $H$ 的线长为
\begin{equation}
wl(G, H) = \min_{\eta} wl(G, H, \eta)
\end{equation}
从 $G$ 到 $H$ 的\emph{线长问题}(Wirelength Problem)
即寻找 $wl(G, H)$ 和能达到最小线长的单射函数 $\eta$。

本文所研究的问题为从超立方体到环嵌入的线长问题,
即对 $\forall d > 2$,寻找 $wl(Q_d, C_{2^d})$ 和能达到最小线长的双射函数 $\eta$。
注意 $\eta$ 在目标图为链或环时也被称为\emph{编号方式}(Numbering),
因为链和环的顶点集都是由自然数构成。

在下文中,若无特殊说明,$n = 2^d$,
$V$、$E$、$\partial$ 等符号都特指超立方体的相应元素。
大多数情况下,读者应该都能够根据上下文判断出符号的意义。

\section{格雷码编号方式}
\label{Section 2.3}

\emph{格雷码}(Graycode)在计算机组成原理、体系结构以及通信领域中有着重要的作用,
并且研究人员已证明(或相信)
格雷码编号方式是从超立方体到链、环、圆柱等结构的嵌入的线长问题的解
\cite{Harper.1964,Guu.1997,Manuel.2011}。

对于从超立方体到环的嵌入,
格雷码编号方式 $\mathcal{G} \colon V_{Q_d} \rightarrow V_{C_{2^d}}$ 有如下的递归定义:
\begin{enumerate}[(1)]
	\item 首先我们有 $Q_1$ 上的格雷码编号方式
		$\mathcal{G} \colon V_{Q_1} \rightarrow V_{C_2}$ 为
		\begin{equation}
		\mathcal{G}(0) = 0, \qquad \mathcal{G}(1) = 1
		\end{equation}
	\item 如果给定了 $Q_d$ 上的格雷码编号方式
		$\mathcal{G} \colon V_{Q_d} \rightarrow V_{C_{2^d}}$($d \ge 1$),
		那么我们给 $Q_d$ 的每一个顶点添加一个坐标前缀 $0$ 和一个相邻的顶点。
		这些新的顶点则给出了一个 $Q_d$ 的拷贝,
		我们为其添加一个坐标前缀 $1$。
		这个新的 $Q_d$ 的编号与原先的相反,
		即与编号为 $2^d - 1$ 的顶点相邻的顶点编号为 $2^d$、
		与编号为 $2^d - 2$ 的顶点相邻的顶点编号为 $2^d + 1$、……、
		与编号为 $0$ 的顶点相邻的顶点编号为 $2^{d + 1} - 1$。
		这样我们便得到了 $Q_{d + 1}$ 上的格雷码编号方式了。
\end{enumerate}
值得注意的是,格雷码编号方式有一个显著的特点:
$C_{2^d}$ 中任意两个相邻的顶点对应在 $Q_d$ 中都有边相连。

\section{立方集}
\label{Section 2.4}

一个 $d$ 阶立方的 $c$ 阶\emph{子立方}(Subcube)是 $Q_d$ 的子图,
导出自一个包含所有在 $d − c$ 个坐标下有相同(固定)值的顶点的集合。
例如,图 \ref{Figure 2-1} 所展示的 $Q_3$ 有 $8$ 个 $2$ 阶子立方(对应立方体有 $8$ 个面)
和 $12$ 个 $1$ 阶子立方(对应立方体有 $12$ 条边)。

一个 $d$ 阶立方的 $c$ 阶子立方的\emph{邻居}(Neighbor)
是在 $d − c$ 个固定坐标下有正好一个坐标不同的任意 $c$ 阶子立方。
我们再以图 \ref{Figure 2-1} 所展示的 $Q_3$ 为例,
对于 $Q_3$ 中的每一个 $2$ 阶子立方,
它的邻居是可以与它一起构成 $Q_3$ 的任意 $2$ 阶子立方,
因此 $Q_3$ 中的每一个 $2$ 阶子立方都只有 $1$ 个邻居;
而对于 $Q_3$ 中的每一个 $1$ 阶子立方,
它的邻居是可以与它一起构成一个 $2$ 阶子立方的任意 $1$ 阶子立方,
因此 $Q_3$ 中的每一个 $1$ 阶子立方都有 $2$ 个邻居。

$Q_d$ 中如果 $S \subseteq V$ 是一个互不相交的不同阶子立方的并集,
并且其中每个 $c_i$ 阶子立方都满足:
对于其他所有 $c_j$ 阶子立方($c_j > c_i$),
该 $c_i$ 阶子立方处在该 $c_j$ 阶子立方的一个邻居中,
那么 $S$ 就被称为\emph{立方集}(Cubal Set)或\emph{复合集}(Composite Set)。
关于立方集更详细的介绍,请参考 \cite{Harper.2004}。
