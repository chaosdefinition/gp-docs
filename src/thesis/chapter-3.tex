%% LaTeX source of Chapter 3 of the thesis.
%% NEVER compile this file. Complie 'thesis.tex' instead.

\chapter{历史研究}
\label{Chapter 3}

\section{边等周问题}
\label{Section 3.1}

在图论中,大量的问题都可以被转化为边等周问题并得到解决,包括一些图嵌入问题。
在 L. H. Harper 的著作 \cite{Harper.1964,Harper.2004} 中,
他通过一系列的转化,利用边等周问题证明了从超立方体到链嵌入的线长问题。

下面我们给出边等周问题的定义,
再简要概述如何利用其解决从超立方体到链嵌入的线长问题,
并给出 $Q_d$ 上的边等周问题证明过程的改进。
注意这种转化方式为我们提供了一种重要的问题解决思路。

\subsection{定义}
\label{Subsection 3.1.1}

对于任意的图 $G$ 和 $S \subseteq V_G$,令
\begin{equation}
\Theta(S) = \{e \in E_G \colon \partial_G(e) = \{v, w\}, v \in S, w \notin S\}
\end{equation}
并称之为 $S$ 的\emph{边界}(Edge-boundary)。
那么对于一个给定的图 $G$ 和自然数 $k$,
\emph{边等周问题}(Edge-Isoperimetric Problem, EIP)
是对于所有的 $S \subseteq V_G$ 并且 $|S| = k$,寻找
\begin{equation}
\min_{|S| = k} |\Theta(S)|
\end{equation}
和能达到该最小值的子集 $S$。

为了便于下文叙述,我们在这里定义一个与 EIP 类似的问题,
并给出其在 $Q_d$ 下与 EIP 的关系。

对于任意的图 $G$ 和 $S \subseteq V_G$,令
\begin{equation}
E(S) = \{e \in E_G \colon \partial_G(e) = \{v, w\}, v \in S, w \in S\}
\end{equation}
并称之为 $S$ 的\emph{导出边集}(Induced Edges)。
对于一个给定的图 $G$ 和自然数 $k$,
\emph{导出边问题}(Induced Edge Problem)
是对于所有的 $S \subseteq V_G$ 并且 $|S| = k$,寻找
\begin{equation}
\max_{|S| = k} |E(S)|
\end{equation}
和能达到该最大值的子集 $S$。

\begin{lemma}
\label{Lemma 3.1}
如果图 $G = (V_G, E_G, \partial_G)$ 是一个 $\delta$ 度正则图,
那么对于 $\forall S \subseteq V_G$,都有
\begin{equation}
|\Theta(S)| + 2 |E(S)| = \delta |S|
\end{equation}
\end{lemma}

\begin{proof}[证明]
$\delta |S|$ 表示 $S$ 对应的边数,然而出现在 $E(S)$ 中的边被计算了两次。
\end{proof}

根据引理 \ref{Lemma 3.1},我们可以知道 $Q_d$ 上的导出边问题与 EIP 等价,
因为对于 $\forall S \subseteq V$ 和 $\forall k$,
$\min_{|S| = k} |\Theta(S)| = k d − 2 \max_{|S| = k} |E(S)|$。

\subsection{线长问题到 EIP}
\label{Subsection 3.1.2}

回想一下,
对于给定的编号方式 $\eta \colon V_{Q_d} \rightarrow V_{P_n}$,
在 $\eta$ 下从 $Q_d$ 到 $P_n$ 的线长 $wl(Q_d, P_n, \eta)$ 为
\begin{equation}
wl(Q_d, P_n, \eta) = \sum_{\substack{
	e \in E \\
	\partial(e) = {v, w}
}} d_{P_n}(\eta(v), \eta(w)) = \sum_{\substack{
	e \in E \\
	\partial(e) = {v, w}
}} |\eta(v) - \eta(w)|
\end{equation}
而从 $Q_d$ 到 $P_n$ 的线长问题则是寻找
\begin{equation}
wl(Q_d, P_n) = \min_\eta wl(Q_d, P_n, \eta)
\end{equation}

为了将从 $Q_d$ 到 $P_n$ 的线长问题转化为 $Q_d$ 上的 EIP,首先我们令
\begin{equation}
S_k(\eta) = \eta^{-1}(\{0, 1, \dots, k - 1\}) = \{v \in V \colon \eta(v) \le k\}
\end{equation}
即编号方式 $\eta$ 下的前 $k$ 个顶点的集合。
接下来,我们令
\begin{equation}
\chi(e, k) = \begin{cases}
	1 & \text{如果}\ \partial(e) = \{v, w\}, \eta(v) \le k < \eta(w) \\
	0 & \text{其他情况}
\end{cases}
\end{equation}
那么我们就有
\begin{align}
wl(Q_d, P_n, \eta) & = \sum_{\substack{
			       e \in E \\
			       \partial(e) = \{v, w\}
		       }} |\eta(v) - \eta(w)|
		     = \sum_{e \in E} \sum_{k = 0}^n \chi(e, k) \\
		   & = \sum_{k = 0}^n \sum_{e \in E} \chi(e, k)
		     = \sum_{k = 0}^n |\Theta(S_k(\eta))|
\end{align}
因此,
\begin{align}
wl(Q_d, P_n) & = \min_{\eta} wl(Q_d, P_n, \eta) \\
	     & = \min_{\eta} \sum_{k = 0}^n |\Theta(S_k(\eta))| \\
	     & \ge \sum_{k = 0}^n \min_{\eta} |\Theta(S_k(\eta))|
	       = \sum_{k = 0}^n \min_{|S| = k} |\Theta(S)|
\end{align}
根据该式,我们可以得到一个结论。

\begin{theorem}
\label{Theorem 3.1}
对于 $\forall \eta \colon V_{Q_d} \rightarrow V_{P_n}$,
如果它的所有初始段 $S_k(\eta)$($0 \le k \le n$)都是 EIP 的解,
那么它本身就是从 $Q_d$ 到 $P_n$ 的线长问题的一个解。
\end{theorem}

根据定理 \ref{Theorem 3.1},我们将问题转化为了 EIP。

\subsection{EIP 证明的改进}
\label{Subsection 3.1.3}

对于 $Q_d$ 上的 EIP,
L. H. Harper 在一篇论文 \cite{Harper.1964} 中提出了下面这个定理,
并给出了他的证明。

\begin{theorem}
\label{Theorem 3.2}
$S \subseteq V_{Q_d}$ 在基数为 $k$ 时有最大的 $|E(S)|$,当且仅当 $S$ 是立方集。
\end{theorem}

在定理 \ref{Theorem 3.2} 的证明过程中,
Harper 对 $d$ 进行数学归纳,并用到了下面这个性质:
如果 $2^{d - 1} \le k < 2^d$,那么
\begin{equation}
E(k + 1) - E(k) = E(k - 2^{d - 1} + 1) - E(k - 2^{d - 1}) + 1
\end{equation}
其中 $E(k)$ 表示一个 $k$-立方集 $S$ 的导出边数 $|E(S)|$,
因为 $|E(S)|$ 并非由 $d$ 决定,而仅仅由 $k = |S|$ 决定。

在这篇论文发表两年后,
A. J. Bernstein 发表了一篇后续论文 \cite{Bernstein.1967},
指出 Harper 忽视了一种情况。
他在这篇后续论文中提出了下面这个引理,用该引理修补了 Harper 的证明中的漏洞:

\begin{lemma}
\label{Lemma 3.2}
对于 $\forall d$ 和 $\forall k, t > 0$ 使得 $k + t < 2^d$,都有
\begin{equation}
E(t) < E(k + t) − E(k) < E(2^d) − E(2^d − t)
\end{equation}
\end{lemma}

在引理 \ref{Lemma 3.2} 的证明过程中,Bernstein 对 $d$ 进行数学归纳,
然后分别考虑了三种情况:
\begin{enumerate}[(1)]
	\item 当 $k \ge 2^{d − 1}$ 时,左右两边不等式成立;
	\item 当 $k + t \le 2^{d − 1}$ 时,左右两边不等式成立;
	\item 当 $k < 2^{d - 1} < k + t$ 时,左右两边不等式成立。
\end{enumerate}

然而我们发现第 3 种情况的证明依然存在纰漏:
Bernstein 并未考虑 $t \ge 2^{d − 1}$ 的情况,
即当 $k < 2^{d − 1} < k + t$ 时,
\begin{equation}
E(2^{d − 1}) − E(k) > E(t) − E(t − (2^{d − 1} − k))
\end{equation}
仅对 $t < 2^{d - 1}$ 成立。
下面我们给出经过补充的情况 3 的完整证明。

\begin{proof}[引理 \ref{Lemma 3.2} 情况 3 的完整证明]
当 $k < 2^{d − 1} < k + t$ 时,
\begin{enumerate}[(1)]
	\item 如果 $t \ge 2^{d - 1}$,那么对于左边的不等式,我们有
		\begin{align}
		E(k + t) - E(k) & = \left[E(k + t) - E(2^{d - 1})\right] +
				    \left[E(2^{d - 1}) - E(k)\right] \\
				& = \left[E(k + t - 2^{d - 1}) + k + t - 2^{d - 1}\right] +
				    \left[E(2^{d - 1}) - E(k)\right] \\
				& = \left[E(k + t - 2^{d - 1}) - E(k)\right] +
				    E(2^{d - 1}) + k + t - 2^{d - 1} \\
				& > E(t - 2^{d - 1}) + E(2^{d - 1}) + t - 2^{d - 1} + k \\
				& = E(t) + k > E(t)
		\end{align}
		对于右边的不等式,我们有
		\begin{align}
		E(k + t) - E(k) & = \left[E(k + t) - E(2^{d - 1})\right] +
				    \left[E(2^{d - 1}) - E(k)\right] \\
				& = \left[E(k + t - 2^{d - 1}) + k + t - 2^{d - 1}\right] +
				    \left[E(2^{d - 1}) - E(k)\right] \\
				& = \left[E(k + t - 2^{d - 1}) - E(k)\right] +
				    E(2^{d - 1}) + k + t - 2^{d - 1} \\
				& < E(2^{d - 1}) - E(2^d - t) + E(2^{d - 1}) +
				    k + t - 2^{d - 1} \\
				& = 2 E(2^{d - 1}) + k + t - 2^{d - 1} - E(2^d - t) \\
				& < 2 E(2^{d - 1}) + 2^{d - 1} - E(2^d - t) \\
				& = E(2^d) - E(2^d - t)
		\end{align}
	\item 相应地,如果 $t < 2^{d - 1}$,那么对于左边的不等式,我们有
		\begin{align}
		E(k + t) − E(k) & = \left[E(k + t) − E(2^{d − 1})\right] +
				    \left[E(2^{d − 1}) - E(k)\right] \\
				& = \left[E(k + t - 2^{d - 1}) + k + t - 2^{d - 1}\right] +
				    \left[E(2^{d - 1}) - E(k)\right] \\
				& > E(k + t - 2^{d - 1}) +
				    \left[E(t) − E(t − (2^{d − 1} − k))\right] \\
				& = E(t)
		\end{align}
		对于右边的不等式,我们有
		\begin{align}
		E(k + t) - E(k) & = \left[E(k + t) - E(2^{d - 1})\right] +
				    \left[E(2^{d - 1}) - E(k)\right] \\
				& = \left[E(k + t - 2^{d - 1}) + k + t - 2^{d - 1}\right] +
				    \left[E(2^{d - 1}) - E(k)\right] \\
				& = \left[E(k + t - 2^{d - 1}) - E(k)\right] +
				    E(2^{d - 1}) + k + t - 2^{d - 1} \\
				& < E(2^{d - 1}) - E(2^{d - 1} - t) + t + k - 2^{d - 1} \\
				& = E(2^d) - E(2^d - t) + k - 2^{d - 1} \\
				& < E(2^d) - E(2^d - t)
		\end{align}
\end{proof}

\section{派生网络}
\label{Section 3.2}

从超立方体到链嵌入的线长问题可以通过 EIP 得到解决,那么从超立方体到环的嵌入呢?
1997 年,Ching-Jung Guu 在她的博士学位论文 \cite{Guu.1997} 中
宣称从超立方体到环嵌入的线长问题已经得到了解决。
她通过创建派生网络这一概念将该问题转化为了 EIP,
并得出了格雷码嵌入有最小的环形线长这一结论。

下面我们给出派生网络及其一系列附加概念的定义,
再简要概述如何利用派生网络将从超立方体到环嵌入的线长问题转化为 EIP,
最后我们简述 Guu 的证明思路,并援引 Jason Erbele 等人的文章 \cite{Erbele.2003},
这篇文章指出了 Guu 的证明过程中的错误。

\subsection{定义}
\label{Subsection 3.2.1}

一个\emph{派生网络}(Derived Network)是一个定义在 $Q_d$ 上的顶点加权图
$D = (V_D, E_D, \partial_D)$,其中 $V_D$ 由 $V_{Q_d}$ 的所有基数为 $2^{d - 1}$ 的子集构成。
对于 $e \in E_D$ 和 $V, W \in V_D$,$\partial_D(e) = \{V, W\}$ 当且仅当 $V \Delta W = 2$,
即 $|V \cap W| = 2^{d - 1} - 1$。
对于 $D$ 中的顶点 $V \in V_D$,它的权重为 $|\Theta(V)|$。

一条派生网络中的路径
$\mathcal{P} = (\mathcal{P}_1, \mathcal{P}_2, \dots, \mathcal{P}_{2^{d - 1}}, \mathcal{P}_{2^{d - 1} + 1})$
是 $D$ 中的一条长为 $2^{d - 1}$ 的路径,
我们用 $2^{d - 1} + 1$ 个 $V_D$ 中的顶点(即 $\mathcal{P}_i$)来表示,
其中相邻两个顶点之间都有边相连。
Guu 将 $\mathcal{P}$ 设计为对应一个编号方式 $\eta_\mathcal{P} \colon V_{Q_d} \rightarrow V_{C_n}$,
其中
\begin{align}
\eta_\mathcal{P}(\mathcal{P}_1 - \mathcal{P}_2) & = 0, & \quad
\eta_\mathcal{P}(\mathcal{P}_2 - \mathcal{P}_1) & = 2^{d - 1}, \\
\eta_\mathcal{P}(\mathcal{P}_2 - \mathcal{P}_3) & = 1, & \quad
\eta_\mathcal{P}(\mathcal{P}_3 - \mathcal{P}_2) & = 2^{d - 1} + 1, \\
						& \cdots & \quad & \cdots \notag \\
\eta_\mathcal{P}(\mathcal{P}_{2^{d - 1} - 1} - \mathcal{P}_{2^{d - 1}}) & = 2^{d - 1} - 2, & \quad
\eta_\mathcal{P}(\mathcal{P}_{2^{d - 1}} - \mathcal{P}_{2^{d - 1} - 1}) & = 2^d - 2, \\
\eta_\mathcal{P}(\mathcal{P}_{2^{d - 1}} - \mathcal{P}_{2^{d - 1} + 1}) & = 2^{d - 1} - 1, & \quad
\eta_\mathcal{P}(\mathcal{P}_{2^{d - 1} + 1} - \mathcal{P}_{2^{d - 1}}) & = 2^d - 1
\end{align}
这样一来,我们便有
\begin{align}
	      \mathcal{P}_1 & = \eta_\mathcal{P}^{-1}(\{0, 1, \dots, 2^{d - 1} - 1\}), \\
	      \mathcal{P}_2 & = \eta_\mathcal{P}^{-1}(\{1, 2, \dots, 2^{d - 1}\}), \\
	      \mathcal{P}_3 & = \eta_\mathcal{P}^{-1}(\{2, 3, \dots, 2^{d - 1} + 1\}), \\
			    & \cdots \notag \\
\mathcal{P}_{2^{d - 1} - 1} & = \eta_\mathcal{P}^{-1}(\{2^{d - 1} - 2, 2^{d - 1} - 1, \dots, 2^d - 3\}), \\
    \mathcal{P}_{2^{d - 1}} & = \eta_\mathcal{P}^{-1}(\{2^{d - 1} - 1, 2^{d - 1}, \dots, 2^d - 2\}), \\
\mathcal{P}_{2^{d - 1} + 1} & = \eta_\mathcal{P}^{-1}(\{2^{d - 1}, 2^{d - 1} + 1, \dots, 2^d - 1\})
\end{align}
我们注意到 $\mathcal{P}_{2^{d - 1} + 1}$ 是 $\mathcal{P}_1$ 在 $V_{Q_d}$ 中的补集,
因此我们也将 $\mathcal{P}_{2^{d - 1} + 1}$ 标记为 $\mathcal{P}_1^C$。

给定任意的 $S \subseteq V_{Q_d}$,
考虑 $Q_d$ 的所有 $d - 1$ 阶子立方,
我们一共有 $2 d$ 个这样的子立方,并将它们标记为 $H_i$,
那么我们定义 $S$ 的\emph{类型}(Type)为
\begin{equation}
Type(S) = \min_{1 \le i \le 2 d} |S \cap H_i|
\end{equation}
路径 $\mathcal{P}$ 的\emph{类型序列}(Type Sequence)则为
\begin{equation}
T(\mathcal{P}) = \left(Type(\mathcal{P}_1), Type(\mathcal{P}_2), \dots, Type(\mathcal{P}_{2^{d - 1}}), Type(\mathcal{P}_1^C)\right)
\end{equation}
注意 $Type(S) \in \{0, 1, \dots, 2^{d - 2}\}$。
Guu 把 $Type(S)$ 在 $0$ 和 $2^{d - 3}$ 之间的子集 $S$ 称为\emph{小集}(Small),
把其余的称为\emph{大集}(Big)。

\subsection{派生网络到 EIP}
\label{Subsection 3.2.2}

首先我们给出一个等式
\begin{equation}
wl(Q_d, C_n, \eta) = \sum_{i = 0}^{2^{d - 1} - 1}
|\Theta(\eta^{-1}(i, i + 1, \dots, i + (2^{d - 1} - 1)))|
\end{equation}
该等式的证明类似于第 \ref{Subsection 3.1.2} 小节中线长问题到 EIP 的转化,
只不过目标图从 $P_n$ 变为了 $C_n$。

根据上一小节中路径 $\mathcal{P}$ 的定义,我们有
\begin{align}
wl(Q_d, C_n, \eta_\mathcal{P}) & = \sum_{i = 0}^{2^{d - 1} - 1}
				   |\Theta(\eta_\mathcal{P}^{-1}(i, i + 1, \dots, i + (2^{d - 1} - 1)))| \\
			       & = \sum_{i = 1}^{2^{d - 1}} |\Theta(\mathcal{P}_i)|
\end{align}
即派生网络中的一条路径所对应的编号方式的线长等于该路径中所有顶点
(不包含 $\mathcal{P}_1^C$)的权重之和。
因此从超立方体到环嵌入的线长问题就是寻找派生网络中有最小权重之和的路径,
从而该问题被转化为了 EIP。

\subsection{证明与缺陷}
\label{Subsection 3.2.3}

通过派生网络把线长问题转化为 EIP 后,
Guu 对格雷码编号方式有最小的环形线长这一结论进行了证明。

首先,令 $\mathcal{P}_\mathcal{G}$ 表示格雷码编号方式在派生网络中对应的路径,
Guu 计算了 $\mathcal{P}_\mathcal{G}$ 的类型序列
\begin{equation}
T(\mathcal{P}_\mathcal{G}) = (0, 1, \dots, 2^{d - 3}, 2^{d - 3} - 1, \dots, 0, 1, \dots, 2^{d - 3}, 2^{d - 3} - 1, \dots, 0)
\end{equation}
她发现 $\mathcal{P}_\mathcal{G}$ 中每一个顶点都是小集,
并且一共有两个类型为 $2^{d - 3}$,三个类型为 $0$,四个类型为 $i$($0 < i <2^{d - 3}$)。

然后,令 $\theta(d, k)$ 表示 $Q_d$ 上 EIP 的解,即
\begin{equation}
\theta(d, k) = \min_{\substack{
	S \subseteq V_{Q_d} \\
	|S| = k
}} |\Theta(S)|
\end{equation}
她利用 Harper 的结论推导出:对于任意的 $V \in V_D$ 且 $Type(V) = t \le 2^{d - 3}$(即 $V$ 为小集),
\begin{equation}
|\Theta(V)| \ge 2 \theta(d - 1, t) + (2^{d - 1} - 2 t)
\end{equation}
并指出 $\mathcal{P}_\mathcal{G}$ 中的每一个顶点都达到了这个下界。

接下来,她对派生网络中的任意路径 $\mathcal{P}$ 构造了一系列变换,
并说明只要证明了对于任意的 $V \in V_D$ 且 $Type(V) = t \ge 2^{d - 3}$(即 $V$ 为大集),
\begin{equation}
\label{Inequality 3-1}
|\Theta(V)| \ge \frac{3}{4} \cdot 2^d
\end{equation}
就能利用这一系列变换证明
\begin{equation}
wl(Q_d, C_n, \eta_\mathcal{P}) \ge wl(Q_d, C_n, \mathcal{G})
\end{equation}
从而证明格雷码编号方式是最优的。

Guu 的论文的余下篇幅都是在证明不等式 (\ref{Inequality 3-1}) 的成立。
然而,Jason Erbele 等人在 \cite{Erbele.2003} 中指出了她的证明过程中的错误:
在她证明该不等式的最后一步,
存在 $S_1 \cup S_2 = S$ 和 $S_1 \cap S_2 \neq \emptyset$,
她却将 $S_1$ 和 $S_2$ 误以为是 $S$ 的划分,
因而利用了 $|S_1| + |S_2| = |S|$ 这一错误的结论用以完成证明。

\section{边拥塞与划分引理}
\label{Section 3.3}

在图嵌入领域,Paul Manuel 等人提出了另一种用于线长问题的方法,
并且他们用该方法成功地解决了从超立方体到网格的嵌入的线长问题 \cite{Manuel.2009}。
下面我们就简要介绍他们的研究。

给定图 $G$、$H$ 和双射函数 $\eta \colon V_G \rightarrow V_H$,
对于边 $e \in E_H$,我们定义
\begin{equation}
EC_\eta(G, H, e) = \{\{v, w\} \in E_G \colon e \in P(\eta(v), \eta(w))\}
\end{equation}
并称之为 $e$ 的\emph{边拥塞}(Edge Congestion)。
相应地,我们则把 $|EC_\eta(G, H, e)|$ 称为 $e$ 的边拥塞度。
而在 $\eta$ 下从 $G$ 到 $H$ 的边拥塞度为
\begin{equation}
|EC_\eta(G, H)| = \max_{e \in E_H} |EC_\eta(G, H, e)|
\end{equation}
为简洁起见,后文中我们都将用 $EC_\eta(e)$ 来表示 $EC_\eta(G, H, e)$。
注意若有 $E \subseteq E_H$,则 $EC_\eta(E) = \bigcup_{e \in E} EC_\eta(e)$。

下面两个引理在 \cite{Manuel.2009} 中已被证明。
注意 $H$ 中的一个边集被认为是一个\emph{边割集}(Edge Cut),
当从 $H$ 中去掉这些边时会导致 $H$ 不连通。

\begin{lemma}[拥塞引理]
\label{Lemma 3.3}
令图 $G = (V_G, E_G, \partial_G)$ 是一个 $\delta$ 度正则图,
$\eta$ 是从图 $G$ 到图 $H$ 的一个嵌入。
令 $S$ 是 $H$ 的一个边割集,
并且去掉 $S$ 会将 $H$ 分割成 $H_1$ 和 $H_2$ 这两个部分。
令 $G_1 = \eta^{-1}(H_1)$,$G_2 = \eta^{-1}(H_2)$。
另外,$S$ 还需满足下列条件:
\begin{enumerate}[(i)]
\item 对于 $i \in \{1, 2\}$ 和每一条边 $\{v, w\} \in E_{G_i}$,$P(v, w)$ 不在 $S$ 中有边;
\item 对于每一条边 $\{v, w\}$ 满足 $v \in G_1$ 和 $w \in G_2$,$P(v, w)$ 在 $S$ 中只有一条边;
\item $G_1$ 是最优的,即对于任意 $V \subseteq V_G$ 且 $|V| = |V_{G_1}|$,$|E(V_{G_1})| \le |E(V)|$。
\end{enumerate}
那么 $|EC_\eta(S)|$ 是最小的,
即对于所有从 $G$ 到 $H$ 的嵌入 $\eta'$,都有 $|EC_\eta(S)| \le |EC_{\eta'}(S)|$。
\end{lemma}

\begin{lemma}[划分引理]
\label{Lemma 3.4}
令 $\eta$ 是从 $G$ 到 $H$ 的一个嵌入。
若存在 $\{S_1, S_2, \dots, S_p\}$ 是 $E_H$ 的一个划分,
其中每一个 $S_i$ 都是 $H$ 的一个边割集,那么
\begin{equation}
wl(G, H) = wl(G, H, \eta) = \sum_{i = 1}^p |EC_\eta(S_i)|
\end{equation}
\end{lemma}

引理 \ref{Lemma 3.4} 为图嵌入领域中的线长问题提供了一种新的解决思路。
然而它并不适用于从超立方体到环嵌入的线长问题,
因为我们找不到 $E_{C_n}$ 的一个这样的划分。
