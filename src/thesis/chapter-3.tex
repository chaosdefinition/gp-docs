%% LaTeX source of Chapter 3 of the thesis.
%% NEVER compile this file. Complie 'thesis.tex' instead.

\chapter{历史研究}
\label{Chapter 3}

\section{边等周问题}
\label{Section 3.1}

在图论中,大量的问题都可以被转化为边等周问题并得到解决,包括一些图嵌入问题。
在 L. H. Harper 的著作 \cite{Harper.1964,Harper.2004} 中,
他通过一系列的转化,利用边等周问题证明了从超立方体到链嵌入的线长问题。

下面我们给出边等周问题的定义,
再简要概述如何利用其解决从超立方体到链嵌入的线长问题,
并给出 $Q_d$ 上的边等周问题证明过程的改进。
注意这种转化方式为我们提供了一种重要的问题解决思路。

\subsection{定义}
\label{Subsection 3.1.1}

对于任意的图 $G$ 和 $S \subseteq V_G$,令
\begin{equation*}
\Theta(S) = \{e \in E_G \colon \partial_G(e) = \{v, w\}, v \in S, w \notin S\}
\end{equation*}
并称之为 $S$ 的\emph{边界}(Edge-boundary)。
那么对于一个给定的图 $G$ 和自然数 $k$,
\emph{边等周问题}(Edge-Isoperimetric Problem, EIP)
是对于所有的 $S \subseteq V_G$ 并且 $|S| = k$,寻找
\begin{equation*}
\min_{|S| = k} |\Theta(S)|
\end{equation*}
和能达到该最小值的子集 $S$。

为了便于下文叙述,我们在这里定义一个与 EIP 类似的问题,
并给出其在 $Q_d$ 下与 EIP 的关系。

对于任意的图 $G$ 和 $S \subseteq V_G$,令
\begin{equation*}
E(S) = \{e \in E_G \colon \partial_G(e) = \{v, w\}, v \in S, w \in S\}
\end{equation*}
并称之为 $S$ 的\emph{导出边集}(Induced Edges)。
对于一个给定的图 $G$ 和自然数 $k$,
\emph{导出边问题}(Induced Edge Problem)
是对于所有的 $S \subseteq V_G$ 并且 $|S| = k$,寻找
\begin{equation*}
\max_{|S| = k} |E(S)|
\end{equation*}
和能达到该最大值的子集 $S$。

\begin{lemma}
\label{Lemma 3.1}
如果图 $G = (V_G, E_G, \partial_G)$ 是一个 $\delta$ 度正则图,
那么对于 $\forall S \subseteq V_G$,都有
\begin{equation*}
|\Theta(S)| + 2 |E(S)| = \delta |S|
\end{equation*}
\end{lemma}

\begin{proof}[证明]
$\delta |S|$ 表示 $S$ 对应的边数,然而出现在 $E(S)$ 中的边被计算了两次。
\end{proof}

根据引理 \ref{Lemma 3.1},我们可以知道 $Q_d$ 上的导出边问题与 EIP 等价,
因为对于 $\forall S \subseteq V$ 和 $\forall k$,
$\min_{|S| = k} |\Theta(S)| = k d − 2 \max_{|S| = k} |E(S)|$。

\subsection{线长问题到 EIP}
\label{Subsection 3.1.2}

回想一下,
对于给定的编号方式 $\eta \colon V_{Q_d} \rightarrow V_{P_n}$,
在 $\eta$ 下从 $Q_d$ 到 $P_n$ 的线长 $wl(Q_d, P_n, \eta)$ 为
\begin{equation*}
wl(Q_d, P_n, \eta) = \sum_{\substack{
	e \in E \\
	\partial(e) = {v, w}
}} d_{P_n}(\eta(v), \eta(w)) = \sum_{\substack{
	e \in E \\
	\partial(e) = {v, w}
}} |\eta(v) - \eta(w)|
\end{equation*}
而从 $Q_d$ 到 $P_n$ 的线长问题则是寻找
\begin{equation*}
wl(Q_d, P_n) = \min_\eta wl(Q_d, P_n, \eta)
\end{equation*}

为了将从 $Q_d$ 到 $P_n$ 的线长问题转化为 $Q_d$ 上的 EIP,首先我们令
\begin{equation*}
S_k(\eta) = \eta^{-1}(\{0, 1, \dots, k - 1\}) = \{v \in V \colon \eta(v) \le k\}
\end{equation*}
即编号方式 $\eta$ 下的前 $k$ 个顶点的集合。
接下来,我们令
\begin{equation*}
\chi(e, k) = \begin{cases}
	1 & \text{如果}\ \partial(e) = \{v, w\}, \eta(v) \le k < \eta(w) \\
	0 & \text{其他情况}
\end{cases}
\end{equation*}
那么我们就有
\begin{align*}
wl(Q_d, P_n, \eta) & = \sum_{\substack{
			       e \in E \\
			       \partial(e) = \{v, w\}
		       }} |\eta(v) - \eta(w)|
		     = \sum_{e \in E} \sum_{k = 0}^n \chi(e, k) \\
		   & = \sum_{k = 0}^n \sum_{e \in E} \chi(e, k)
		     = \sum_{k = 0}^n |\Theta(S_k(\eta))|
\end{align*}
因此,
\begin{align*}
wl(Q_d, P_n) & = \min_{\eta} wl(Q_d, P_n, \eta) \\
	     & = \min_{\eta} \sum_{k = 0}^n |\Theta(S_k(\eta))| \\
	     & \ge \sum_{k = 0}^n \min_{\eta} |\Theta(S_k(\eta))|
	       = \sum_{k = 0}^n \min_{|S| = k} |\Theta(S)|
\end{align*}
根据该式,我们可以得到一个结论。

\begin{theorem}
\label{Theorem 3.1}
对于 $\forall \eta \colon V_{Q_d} \rightarrow V_{P_n}$,
如果它的所有初始段 $S_k(\eta)$($0 \le k \le n$)都是 EIP 的解,
那么它本身就是从 $Q_d$ 到 $P_n$ 的线长问题的一个解。
\end{theorem}

根据定理 \ref{Theorem 3.1},我们将问题转化为了 EIP。

\subsection{EIP 证明的改进}
\label{Subsection 3.1.3}

对于 $Q_d$ 上的 EIP,
L. H. Harper 在一篇论文 \cite{Harper.1964} 中提出了下面这个定理,
并给出了他的证明。

\begin{theorem}
\label{Theorem 3.2}
$S \subseteq V_{Q_d}$ 在基数为 $k$ 时有最大的 $|E(S)|$,当且仅当 $S$ 是立方集。
\end{theorem}

在定理 \ref{Theorem 3.2} 的证明过程中,
Harper 对 $d$ 进行数学归纳,并用到了下面这个性质:
如果 $2^{d - 1} \le k < 2^d$,那么
\begin{equation*}
E(k + 1) - E(k) = E(k - 2^{d - 1} + 1) - E(k - 2^{d - 1}) + 1
\end{equation*}
其中 $E(k)$ 表示一个 $k$-立方集 $S$ 的导出边数 $|E(S)|$,
因为 $|E(S)|$ 并非由 $d$ 决定,而仅仅由 $k = |S|$ 决定。

在这篇论文发表两年后,
A. J. Bernstein 发表了一篇后续论文 \cite{Bernstein.1967},
指出 Harper 忽视了一种情况。
他在这篇后续论文中提出了下面这个引理,用该引理修补了 Harper 的证明中的漏洞:

\begin{lemma}
\label{Lemma 3.2}
对于 $\forall d$ 和 $\forall k, t > 0$ 使得 $k + t < 2^d$,都有
\begin{equation*}
E(t) < E(k + t) − E(k) < E(2^d) − E(2^d − t)
\end{equation*}
\end{lemma}

在引理 \ref{Lemma 3.2} 的证明过程中,Bernstein 对 $d$ 进行数学归纳,
然后分别考虑了三种情况:
\begin{enumerate}[(1)]
	\item 当 $k \ge 2^{d − 1}$ 时,左右两边不等式成立;
	\item 当 $k + t \le 2^{d − 1}$ 时,左右两边不等式成立;
	\item 当 $k < 2^{d - 1} < k + t$ 时,左右两边不等式成立。
\end{enumerate}

然而我们发现第 3 种情况的证明依然存在纰漏:
Bernstein 并未考虑 $t \ge 2^{d − 1}$ 的情况,
即当 $k < 2^{d − 1} < k + t$ 时,
\begin{equation*}
E(2^{d − 1}) − E(k) > E(t) − E(t − (2^{d − 1} − k))
\end{equation*}
仅对 $t < 2^{d - 1}$ 成立。
下面我们给出经过补充的情况 3 的完整证明。

\begin{proof}[引理 \ref{Lemma 3.2} 情况 3 的完整证明]
当 $k < 2^{d − 1} < k + t$ 时,
\begin{enumerate}[(1)]
	\item 如果 $t \ge 2^{d - 1}$,那么对于左边的不等式,我们有
		\begin{align*}
		E(k + t) - E(k) & = \left[E(k + t) - E(2^{d - 1})\right] +
				    \left[E(2^{d - 1}) - E(k)\right] \\
				& = \left[E(k + t - 2^{d - 1}) + k + t - 2^{d - 1}\right] +
				    \left[E(2^{d - 1}) - E(k)\right] \\
				& = \left[E(k + t - 2^{d - 1}) - E(k)\right] +
				    E(2^{d - 1}) + k + t - 2^{d - 1} \\
				& > E(t - 2^{d - 1}) + E(2^{d - 1}) + t - 2^{d - 1} + k \\
				& = E(t) + k > E(t)
		\end{align*}
		对于右边的不等式,我们有
		\begin{align*}
		E(k + t) - E(k) & = \left[E(k + t) - E(2^{d - 1})\right] +
				    \left[E(2^{d - 1}) - E(k)\right] \\
				& = \left[E(k + t - 2^{d - 1}) + k + t - 2^{d - 1}\right] +
				    \left[E(2^{d - 1}) - E(k)\right] \\
				& = \left[E(k + t - 2^{d - 1}) - E(k)\right] +
				    E(2^{d - 1}) + k + t - 2^{d - 1} \\
				& < E(2^{d - 1}) - E(2^d - t) + E(2^{d - 1}) +
				    k + t - 2^{d - 1} \\
				& = 2 E(2^{d - 1}) + k + t - 2^{d - 1} - E(2^d - t) \\
				& < 2 E(2^{d - 1}) + 2^{d - 1} - E(2^d - t) \\
				& = E(2^d) - E(2^d - t)
		\end{align*}
	\item 相应地,如果 $t < 2^{d - 1}$,那么对于左边的不等式,我们有
		\begin{align*}
		E(k + t) − E(k) & = \left[E(k + t) − E(2^{d − 1})\right] +
				    \left[E(2^{d − 1}) - E(k)\right] \\
				& = \left[E(k + t - 2^{d - 1}) + k + t - 2^{d - 1}\right] +
				    \left[E(2^{d - 1}) - E(k)\right] \\
				& > E(k + t - 2^{d - 1}) +
				    \left[E(t) − E(t − (2^{d − 1} − k))\right] \\
				& = E(t)
		\end{align*}
		对于右边的不等式,我们有
		\begin{align*}
		E(k + t) - E(k) & = \left[E(k + t) - E(2^{d - 1})\right] +
				    \left[E(2^{d - 1}) - E(k)\right] \\
				& = \left[E(k + t - 2^{d - 1}) + k + t - 2^{d - 1}\right] +
				    \left[E(2^{d - 1}) - E(k)\right] \\
				& = \left[E(k + t - 2^{d - 1}) - E(k)\right] +
				    E(2^{d - 1}) + k + t - 2^{d - 1} \\
				& < E(2^{d - 1}) - E(2^{d - 1} - t) + t + k - 2^{d - 1} \\
				& = E(2^d) - E(2^d - t) + k - 2^{d - 1} \\
				& < E(2^d) - E(2^d - t)
		\end{align*}
\end{proof}

\section{Guu 的研究与缺陷}
\label{Section 3.2}

待撰写。

\section{边拥塞与分区引理}
\label{Section 3.3}

待撰写。
