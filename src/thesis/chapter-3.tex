%% LaTeX source of Chapter 3 of the thesis.
%% NEVER compile this file. Complie 'thesis.tex' instead.

\chapter{历史研究}
\label{Chapter 3}

\section{边等周问题}
\label{Section 3.1}

在图论中,大量的问题都可以被转化为边等周问题并得到解决,包括一些图嵌入问题。
在 L. H. Harper 的著作 \cite{Harper.1964,Harper.2004} 中,
他通过一系列的转化,利用边等周问题证明了从超立方体到链嵌入的线长问题。

下面我们给出边等周问题的定义,
再简要概述如何利用其解决从超立方体到链嵌入的线长问题,
并给出 $Q_d$ 上的边等周问题证明过程的改进。
注意这种转化方式为我们提供了一种重要的问题解决思路。

\subsection{定义}
\label{Subsection 3.1.1}

对任意的图 $G$ 和 $S \subseteq V_G$,我们定义:
\begin{equation*}
\Theta(S) = \{e \in E_G \colon \partial_G(e) = \{v, w\}, v \in S, w \notin S\}
\end{equation*}
并称之为 $S$ 的边界(Edge-boundary)。
那么对于一个给定的图 $G$ 和自然数 $k$,
边等周问题(Edge-Isoperimetric Problem, EIP)
是对于所有的 $S \subseteq V_G$ 并且 $|S| = k$,
寻找
\begin{equation*}
\min_{|S| = k} |\Theta(S)|
\end{equation*}
和能达到该最小值的子集 $S$。

\subsection{线长问题到 EIP}
\label{Subsection 3.1.2}

回想一下,
对于给定的编号方式 $\eta \colon V_{Q_d} \rightarrow V_{P_n}$,
在 $\eta$ 下从 $Q_d$ 到 $P_n$ 的线长 $wl(Q_d, P_n, \eta)$ 为
\begin{equation*}
wl(Q_d, P_n, \eta) = \sum_{\substack{
	e \in E \\
	\partial(e) = {v, w}
}} |P_{P_n}(\eta(v), \eta(w))| = \sum_{\substack{
	e \in E \\
	\partial(e) = {v, w}
}} |\eta(v) - \eta(w)|
\end{equation*}
而从 $Q_d$ 到 $P_n$ 的线长问题则是寻找
\begin{equation*}
wl(Q_d, P_n) = \min_\eta wl(Q_d, P_n, \eta)
\end{equation*}

为了将从 $Q_d$ 到 $P_n$ 的线长问题转化为 EIP,首先我们令
\begin{equation*}
S_k(\eta) = \eta^{-1}(\{0, 1, \dots, k - 1\}) = \{v \in V \colon \eta(v) \le k\}
\end{equation*}
即编号方式 $\eta$ 下的前 $k$ 个顶点的集合。
接下来,我们令
\begin{equation*}
\chi(e, k) = \begin{cases}
	1 & \text{如果}\ \partial(e) = \{v, w\}, \eta(v) \le k < \eta(w) \\
	0 & \text{其他情况}
\end{cases}
\end{equation*}
那么我们就有
\begin{align*}
wl(Q_d, P_n, \eta) & = \sum_{\substack{
			       e \in E \\
			       \partial(e) = \{v, w\}
		       }} |\eta(v) - \eta(w)|
		     = \sum_{e \in E} \sum_{k = 0}^n \chi(e, k) \\
		   & = \sum_{k = 0}^n \sum_{e \in E} \chi(e, k)
		     = \sum_{k = 0}^n |\Theta(S_k(\eta))|
\end{align*}
因此,
\begin{align*}
wl(Q_d, P_n) & = \min_{\eta} wl(Q_d, P_n, \eta) \\
	     & = \min_{\eta} \sum_{k = 0}^n |\Theta(S_k(\eta))| \\
	     & \ge \sum_{k = 0}^n \min_{\eta} |\Theta(S_k(\eta))|
	       = \sum_{k = 0}^n \min_{|S| = k} |\Theta(S)|
\end{align*}
根据该式,我们可以得到一个结论。

\begin{theorem}
\label{Theorem 3.1}
对于 $\forall \eta \colon V_{Q_d} \rightarrow V_{P_n}$,
如果它的所有初始段 $S_k(\eta)$($0 \le k \le n$)都是 EIP 的解,
那么它本身就是从 $Q_d$ 到 $P_n$ 的线长问题的一个解。
\end{theorem}

根据定理 \ref{Theorem 3.1},我们将问题转化为了 EIP。

\subsection{证明的改进}
\label{Subsection 3.1.3}
