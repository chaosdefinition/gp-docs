%% LaTeX source of Chapter 5 of the thesis.
%% NEVER compile this file. Complie 'thesis.tex' instead.

\chapter{结果分析}
\label{Chapter 5}

利用在第 \ref{Chapter 4} 章构建好的分布式计算程序 Hyperspark,
我们对 $3$ 阶和 $4$ 阶超立方体到环的嵌入进行了计算。

\section{示例说明}
\label{Section 5.1}

我们在前文中提到过,
一个从超立方体到环的编号方式可以被视为一个 $0$ 到 $2^d - 1$ 这 $2^d$ 个数的排列。
因此下文中给出的所有结果都将以排列的形式展现出来。
在这里我们以格雷码编号方式 $\mathcal{G} \colon V_{Q_3} \rightarrow V_{C_8}$ 为例,
将 $\mathcal{G}$ 表示为
\begin{equation*}
\mathcal{G} = [0, 1, 3, 2, 6, 7, 5, 4]
\end{equation*}
其中每一个数是 $Q_3$ 上每一个顶点的十进制表示形式。
该排列表示 $Q_3$ 中的顶点 $000$(十进制表示为 $0$)映射到 $C_8$ 中的顶点 $0$(第一个位置)、
$Q_3$ 中的顶点 $001$(十进制表示为 $1$)映射到 $C_8$ 中的顶点 $1$(第二个位置 )、……、
$Q_3$ 中的顶点 $100$(十进制表示为 $4$)映射到 $C_8$ 中的顶点 $7$(第八个位置 )。

\section{$3$ 阶情况}
\label{Section 5.2}

首先是 $3$ 阶的计算结果。
$3$ 阶的情况下一共只有 $40,320$ 种编号方式,
在如今的计算机的运算能力下仅依靠未做任何优化的单机计算程序(如算法 \ref{Algorithm 4.4})
就可以在瞬间得到结果。
因此,我们很自然地选择了 $3$ 阶情况的计算作为验证 Hyperspark 正确性的测试;
而对于程序的运行时间等其他指标,我们在这次计算中将选择性地忽略。

我们先运行算法 \ref{Algorithm 4.9} 的实现,即 Hyperspark。调用
\begin{equation*}
\textnormal{CalculateMinimumWirelength2}(3, 3);
\end{equation*}
其中输入数据为 $d = 3$ 和 $length = 3$,
表示计算 $3$ 阶的情况,并设定初始排列的长度为 $3$。
初始排列的长度决定了我们要将任务划分成多少份,即要把小任务交给多少个从节点进行计算,
其对应关系为
\begin{equation*}
length \rightarrow \frac{(length - 1)!}{2}
\end{equation*}
因为 $0, 1, 2$ 的不带重复的环形排列只有 $1$ 个,
而要往一个 $0, 1, \dots, x - 1$ 的环形排列中插入 $x$ 则有 $x$ 个不带重复的位置可供插入。

通过上述调用,我们最终得到 $3$ 阶情况下的最短线长为 $20$,以及 $48$ 个 $3$ 阶排列:
\begin{align*}
[0, 6, 4, 7, 5, 3, 1, 2], \quad [0, 6, 4, 5, 7, 3, 1, 2], & \quad [0, 4, 6, 7, 5, 3, 1, 2], \quad [0, 4, 6, 5, 7, 3, 1, 2], \\
[0, 3, 1, 7, 5, 6, 4, 2], \quad [0, 3, 1, 5, 7, 6, 4, 2], & \quad [0, 3, 1, 7, 5, 4, 6, 2], \quad [0, 3, 1, 5, 7, 4, 6, 2], \\
[0, 6, 4, 7, 5, 1, 3, 2], \quad [0, 6, 4, 5, 7, 1, 3, 2], & \quad [0, 4, 6, 7, 5, 1, 3, 2], \quad [0, 4, 6, 5, 7, 1, 3, 2], \\
[0, 4, 5, 1, 7, 3, 6, 2], \quad [0, 4, 5, 1, 3, 7, 6, 2], & \quad [0, 4, 5, 1, 7, 3, 2, 6], \quad [0, 4, 5, 1, 3, 7, 2, 6], \\
[0, 4, 1, 5, 7, 3, 6, 2], \quad [0, 4, 1, 5, 3, 7, 6, 2], & \quad [0, 4, 1, 5, 7, 3, 2, 6], \quad [0, 4, 1, 5, 3, 7, 2, 6], \\
[0, 1, 5, 4, 7, 6, 3, 2], \quad [0, 1, 5, 4, 6, 7, 3, 2], & \quad [0, 1, 4, 5, 7, 6, 3, 2], \quad [0, 1, 4, 5, 6, 7, 3, 2], \\
[0, 1, 3, 7, 5, 6, 4, 2], \quad [0, 1, 3, 5, 7, 6, 4, 2], & \quad [0, 1, 3, 7, 5, 4, 6, 2], \quad [0, 1, 3, 5, 7, 4, 6, 2], \\
[0, 5, 1, 7, 3, 6, 2, 4], \quad [0, 5, 1, 3, 7, 6, 2, 4], & \quad [0, 5, 1, 7, 3, 2, 6, 4], \quad [0, 5, 1, 3, 7, 2, 6, 4], \\
[0, 1, 5, 7, 3, 6, 2, 4], \quad [0, 1, 5, 3, 7, 6, 2, 4], & \quad [0, 1, 5, 7, 3, 2, 6, 4], \quad [0, 1, 5, 3, 7, 2, 6, 4], \\
[0, 1, 3, 2, 7, 6, 5, 4], \quad [0, 1, 3, 2, 6, 7, 5, 4], & \quad [0, 1, 3, 2, 7, 6, 4, 5], \quad [0, 1, 3, 2, 6, 7, 4, 5], \\
[0, 1, 5, 4, 7, 6, 2, 3], \quad [0, 1, 5, 4, 6, 7, 2, 3], & \quad [0, 1, 4, 5, 7, 6, 2, 3], \quad [0, 1, 4, 5, 6, 7, 2, 3], \\
[0, 1, 2, 3, 7, 6, 5, 4], \quad [0, 1, 2, 3, 6, 7, 5, 4], & \quad [0, 1, 2, 3, 7, 6, 4, 5], \quad [0, 1, 2, 3, 6, 7, 4, 5]
\end{align*}
注意我们已经在生成全排列的算法中排除了旋转对称和翻转对称这两种“重复情况”,
因此实际情况中的最优排列数肯定会大于 $48$。
并且我们可以对其进行估算:
长为 $n$ 的全排列有 $2 n$ 种对称情况,正如我们在第 \ref{Subsection 4.3.2} 小节所见;
反过来,知道了不带重复的最优排列数,
我们可以估算总的最优排列数为不带重复的最优排列数乘以 $2 n$。
也就是说我们估算 $3$ 阶情况下的最优排列数应该是 $48 \times 2 \times 8 = 768$。

我们接下来运行算法 \ref{Algorithm 4.4} 的实现。调用
\begin{equation*}
\textnormal{CalculateMinimumWirelength}(3);
\end{equation*}
其中输入数据为 $d = 3$,表示计算 $3$ 阶的情况。

通过上述调用,我们得到 $3$ 阶情况下的最短线长为 $20$,以及 $768$ 个 $3$ 阶排列
(由于数据量过大,我们不在这里列出),从而印证了我们的估算。
同时这也说明了我们 Hyperspark 的正确性,以及在 $3$ 阶情况下 Guu 的结论的正确性。

下面我们对这 $48$ 个排列进行简单的数据分析。
为此我们开发了一个辅助性的 Matlab 画图程序,
用来把每一个排列和它的线长分布情况以图形的方式清晰直观地展现出来。
该程序的核心源代码参见附录 B。
由于篇幅限制,我们从 $48$ 个排列中随机选择了两个排列的图形放在本文中,
一个是格雷码编号方式,见图 \ref{Figure 5-1-a};
另一个则是我们结果集中的第一个编号方式,见图 \ref{Figure 5-1-b}。

\begin{figure}[h!]
	\centering
	\subfloat[格雷码编号方式]{
		\includesvg[width = 0.45\textwidth]{figure-5-1-a}
		\label{Figure 5-1-a}
	}
	\subfloat[另一种最优编号方式]{
		\includesvg[width = 0.45\textwidth]{figure-5-1-b}
		\label{Figure 5-1-b}
	}
	\caption{$Q_3$ 到 $C_{8}$ 的两种编号方式}
	\label{Figure 5-1}
\end{figure}

上面两个图中的黑色的圆表示我们的目标图 $C_8$。
$C_8$ 的顶点从圆上最顶端的位置起沿顺时针方向依次分布在圆上,
我们用一条指向圆心的黑色线段来标识。
对于 $Q_3$,我们用数字 $0, 1, \dots, 7$——即十进制表示形式——来表示它的顶点,
用黑色圆外的每一段彩色圆弧来表示它的每一条边。

注意我们将 $Q_3$ 的每一条边用不同的颜色进行了区分,
而区分的根据则是一条边对应的两个顶点的哪一位不同。
回想一下,在超立方体 $Q_d$ 中,
两个顶点 $v, w \in V_{Q_d}$ 之间有边相连当且仅当 $v$ 和 $w$ 的坐标(二进制表示形式)
正好只有一位不相同。
在这里,我们用红色的圆弧表示第 $1$ 位(最低位)坐标不相同的边,
用绿色的圆弧表示第 $2$ 位(次低位)坐标不相同的边,
接下来是蓝色和粉色(粉色在 $3$ 阶情况中不存在,但存在于接下来的 $4$ 阶情况中)。

通过画图程序可视化所有 $48$ 个排列后,
我们发现 $3$ 阶情况下的最优排列有一个共同的特点:
每一个最优排列都存在 $4$ 条 $C_8$ 中的边,它们的边拥塞度为 $2$,
并且它们的位置也是相对固定的,即若找到一条这样的边,我们可以马上找到其余的三条;
而其他剩下的 $4$ 条边每一条都有边拥塞度大于 $2$。
不仅如此,我们还发现包含边拥塞度为 $2$ 的 $Q_3$ 中的边只对应同一种颜色,
而包含边拥塞度大于 $2$ 的 $Q_3$ 中的边则至少对应两种颜色。

用更严格的形式来表示,对于 $\forall e \in E_{Q_d}$ 且 $\partial(e) = \{v, w\}$,
我们令
\begin{equation*}
t(e) = i,\ \text{其中}\ v_i \neq w_i, 1 \le i \le d
\end{equation*}
表示 $Q_d$ 中的边 $e$ 的类型。
那么对于每一个最优编号方式 $\eta \colon V_{Q_3} \rightarrow V_{C_8}$,
\begin{enumerate}[(1)]
	\item 存在 $4$ 条边 $e_1, e_2, e_3, e_4 \in E_{C_8}$ 满足对于 $\forall k \in \{1, 2, 3, 4\}$,
		\begin{equation}
		\label{Equation 5-1}
		|EC_\eta(e_k)| = 2
		\end{equation}
	\item 对于 $\forall k \in \{1, 2, 3, 4\}$
		和 $\forall e_1', e_2' \in EC_\eta(e_k)$,
		如果同时有 $e_k \in P(e_1')$ 和 $e_k \in P(e_2')$,
		那么
		\begin{equation}
		\label{Equation 5-2}
		t(e_1') = t(e_2')
		\end{equation}
	\item 若令 $e_k = \{i_k, j_k\}$,$1 \le k \le 4$,那么有
		\begin{equation}
		\label{Equation 5-3}
		\begin{cases}
		i_1 \equiv (i_2 - 2) \bmod 8 \equiv (i_3 - 4) \bmod 8 \equiv (i_4 - 6) \bmod 8 \\
		j_1 \equiv (j_2 - 2) \bmod 8 \equiv (j_3 - 4) \bmod 8 \equiv (j_4 - 6) \bmod 8
		\end{cases}
		\end{equation}
\end{enumerate}

我们把式 (\ref{Equation 5-1}) 称为边拥塞公式,
把式 (\ref{Equation 5-2}) 称为同色公式,
把式 (\ref{Equation 5-3}) 称为位置公式。

注意这三个公式只适用于 $Q_3$ 到 $C_8$ 的最优编号方式,
对于这一特性是否在 $4$ 阶情况下也保持成立,我们将在下一节中进行探索。

\section{$4$ 阶情况}
\label{Section 5.3}

下面是 $4$ 阶的计算结果。
$4$ 阶的情况下一共只有 $16 !$ 种编号方式,
因此算法 \ref{Algorithm 4.4} 已不再适用了。
我们直接运行 Hyperspark,调用
\begin{equation*}
\textnormal{CalculateMinimumWirelength2}(4, 4);
\end{equation*}
其中输入数据为 $d = 4$ 和 $length = 4$。
$length = 4$ 的条件下主节点将生成 $\frac{3 !}{2} = 3$ 个索引并分配给 $3$ 个从节点,
每个从节点将各自的索引还原为长度为 $4$ 的不同的初始排列,
再用带剪枝的插入法生成不带重复的全排列。

通过上述调用,我们最终得到 $4$ 阶情况下的最短线长为 $88$
(说明在 $4$ 阶情况下 Guu 的结论也是正确的),
以及来自从节点的 $3$ 个结果集,我们分别称为结果集 $1$、$2$ 和 $3$。
其中结果集 $1$ 和 $3$ 各自包含了 $81,920$ 个最优排列,
结果集 $2$ 则包含了 $229,376$ 个最优排列
(因此我们可以估算总的最优排列数为 $(81920 \times 2 + 229376) \times 2 \times 16 = 12,582,912$,
在这里我们不做验证)。
我们从每个结果集中分别挑选 $1$ 个最优编号方式的图形放在本文中,
外加格雷码编号方式(它存在于结果集 $2$ 中),
见图 \ref{Figure 5-2} 和图 \ref{Figure 5-3}。

\begin{figure}[h!]
	\centering
	\subfloat[格雷码编号方式]{
		\includesvg[width = 0.45\textwidth]{figure-5-2-a}
		\label{Figure 5-2-a}
	}
	\subfloat[结果集 1 中的一个编号方式]{
		\includesvg[width = 0.45\textwidth]{figure-5-2-b}
		\label{Figure 5-2-b}
	}
	\caption{$Q_4$ 到 $C_{16}$ 的两种编号方式}
	\label{Figure 5-2}
\end{figure}

\begin{figure}[h!]
	\centering
	\subfloat[结果集 2 中的一个编号方式]{
		\includesvg[width = 0.45\textwidth]{figure-5-3-a}
		\label{Figure 5-3-a}
	}
	\subfloat[结果集 3 中的一个编号方式]{
		\includesvg[width = 0.45\textwidth]{figure-5-3-b}
		\label{Figure 5-3-b}
	}
	\caption{$Q_4$ 到 $C_{16}$ 的另外两种编号方式}
	\label{Figure 5-3}
\end{figure}

我们惊奇地发现,我们在上一节中提出的特性不但适用于 $3$ 阶情况,
也同样适用于 $4$ 阶情况:
每一个最优排列都存在 $4$ 条 $C_{16}$ 中的边,它们的边拥塞度为 $4$,位置相对固定;
其他剩下的 $12$ 条边每一条都有边拥塞度大于 $4$。
包含边拥塞度为 $4$ 的 $Q_4$ 中的边只对应同一种颜色,
包含边拥塞度大于 $4$ 的 $Q_4$ 中的边至少对应两种颜色。

下面我们给出一个更一般的结论。
\begin{conjecture}
\label{Conjecture 5.1}
对于 $\forall d \ge 2$ 和每一个最优编号方式 $\eta \colon V_{Q_d} \rightarrow V_{C_n}$,
\begin{enumerate}[(1)]
	\item 存在 $4$ 条边 $e_1, e_2, e_3, e_4 \in E_{C_n}$ 满足对于 $\forall k \in \{1, 2, 3, 4\}$,
		\begin{equation}
		\label{Equation 5-4}
		|EC_\eta(e_k)| = 2^{d - 2}
		\end{equation}
	\item 对于 $\forall k \in \{1, 2, 3, 4\}$
		和 $\forall e_1', e_2' \in EC_\eta(e_k)$,
		如果同时有 $e_k \in P(e_1')$ 和 $e_k \in P(e_2')$,
		那么
		\begin{equation}
		\label{Equation 5-5}
		t(e_1') = t(e_2')
		\end{equation}
	\item 若令 $e_k = \{i_k, j_k\}$,$1 \le k \le 4$,那么有
		\begin{equation}
		\label{Equation 5-6}
		\begin{cases}
		i_1 \equiv (i_2 - 2^{d - 2}) \bmod n \equiv (i_3 - 2^{d - 1}) \bmod n \equiv (i_4 - 3 \cdot 2^{d - 2}) \bmod n \\
		j_1 \equiv (j_2 - 2^{d - 2}) \bmod n \equiv (j_3 - 2^{d - 1}) \bmod n \equiv (j_4 - 3 \cdot 2^{d - 2}) \bmod n
		\end{cases}
		\end{equation}
\end{enumerate}
\end{conjecture}

对于该结论是否在 $d \ge 5$ 时成立,我们还无法对其进行验证。
然而我们相信这个结论是正确的,
因为 $5$ 阶格雷码编号方式 $\mathcal{G} \colon V_{Q_5} \rightarrow V_{C_{32}}$
依然满足这三个式子。
