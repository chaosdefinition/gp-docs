%% LaTeX source of Chapter 6 of the thesis.
%% NEVER compile this file. Complie 'thesis.tex' instead.

\chapter{结论}
\label{Chapter 6}

纵观全文,我们在本文中研究了从超立方体到环的嵌入的一些性质。
首先我们介绍了图嵌入领域
——更具体地说,则是从超立方体到链、环等结构的嵌入问题——
的历史研究和进展,
包括由 L. H. Harper 和 A. J. Bernstein 解决的边等周问题 \cite{Harper.1964,Bernstein.1967}、
C. Guu 提出的派生网络 \cite{Guu.1997},
以及 P. Manuel 提出的边拥塞理论 \cite{Manuel.2009} 等,
并对边等周问题的证明提出了改进。
接下来我们引入了分布式计算,
给出了分布式计算程序 Hyperspark 所用到的算法和一些理论基础,
利用工程的方法来验证和探索从超立方体到环嵌入的一些重要性质,
并根据得到的结果集提出了最优嵌入的一些新的特性。

根据我们的结果,Guu 的结论
——即从超立方体到环的嵌入在格雷码编号方式下有最小的线长——
在 $3$ 阶和 $4$ 阶情况下都是正确的。
由于有限的计算资源,以及该问题自身的限制,
我们无法对 $5$ 阶及以上的情况进行验证。
而在新特性方面,利用 Matlab 画图程序对 Hyperspark 的结果集进行数据可视化后,
我们找到了一些最优编号方式所共有的性质,
并对其进行了归纳和总结,见猜想 \ref{Conjecture 5.1}。

\section{未来工作}
\label{Section 6.1}

针对从超立方体到环嵌入这一问题,以及我们的研究结果,
未来还有许多进一步的工作可以继续开展:
\begin{enumerate}[(1)]
	\item 首先,我们可以继续进行图嵌入理论上的研究。
		对于 $5$ 阶及以上的情况,由于巨大的数据量,
		现今任何工程上的方法都不可能一劳永逸地解决该问题。
		我们相信理论上存在一种方法能被给出用来证明该问题,
		亦或是修补 Guu 的证明。
	\item 对于我们提出的猜想 \ref{Conjecture 5.1},我们并没有给出它的证明,
		因此在下一步的工作中我们可以先对该猜想进行证明(或证伪),
		并探究最后的结论对于该问题的解决是否有意义。
	\item 在运行 $4$ 阶情况时,我们的程序产生了大量数据(结果集中的排列),
		然而我们没有对其进行完全分析。
		因此在未来我们可以利用适当的统计分析方法和更加先进的数据分析工具,
		对我们的数据加以详细研究和概括总结。
\end{enumerate}
