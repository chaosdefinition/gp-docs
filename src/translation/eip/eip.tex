%% LaTeX source of EIP.
%% Use command 'xelatex --shell-escape eip.tex' TWICE to compile it into a PDF.

%%%%%%%%%%%%%%%%%%%%%%%%%%%%%%%%%%%%%%%%%%%%%%%%%%%%%%%%%%%%%%%%%%%%%%%%%%%%%%%%
% document preamble %%%%%%%%%%%%%%%%%%%%%%%%%%%%%%%%%%%%%%%%%%%%%%%%%%%%%%%%%%%%
%%%%%%%%%%%%%%%%%%%%%%%%%%%%%%%%%%%%%%%%%%%%%%%%%%%%%%%%%%%%%%%%%%%%%%%%%%%%%%%%
\documentclass[12pt, a4paper]{article}

\usepackage[UTF8]{ctex}		% for Chinese environment
\usepackage[perpage]{footmisc}	% for per page footnotes
\usepackage{enumerate}		% for customized list numbering
\usepackage{hyperref}		% for bookmarks and hyperlinks
\usepackage{amsmath}		% for mathematical features
\usepackage{amssymb}		% for mathematical symbols
\usepackage{amsthm}		% for theorems
\usepackage{svg}		% for svg images
\usepackage{csquotes}		% for customized quotation

\author{L. H. Harper}
\date{2016 年 3 月 11 日}

% hyperref options
\hypersetup{
	% hide links
	hidelinks,
	% number bookmarks
	bookmarksnumbered = true,
	% fit width
	pdfstartview = {FitH},
	% PDF title
	pdftitle = {Chinese Translation of Chapter 1 of Global Methods for Combinatorial Isoperimetric Problems},
	% PDF subject
	pdfsubject = {The Edge-Isoperimetric Problem},
	% PDF author
	pdfauthor = {Zhuojia Shen}
}

% theorem-like environments
\newtheorem{theorem}{定理}
\newtheorem{lemma}{引理}
\newtheorem{corollary}{推论}
\newtheorem{exercise}{练习}
\newtheorem{definition}{定义}
\newtheorem{exercisewithanswer}{练习}

% a new environment 'answer' inheriting 'proof' but without QED symbol
\newenvironment{answer}[1][Answer]{
	% before
	\begin{proof}[#1]
	\let\qed\relax
}{
	% after
	\end{proof}
}

% svg options
\setsvg{
	svgpath = {./images/},		% svg directory
	inkscape = {inkscape -z -D}	% Inkscape command
}

\setlength{\tabcolsep}{4pt}	% space between tabular columns

%%%%%%%%%%%%%%%%%%%%%%%%%%%%%%%%%%%%%%%%%%%%%%%%%%%%%%%%%%%%%%%%%%%%%%%%%%%%%%%%
% document body %%%%%%%%%%%%%%%%%%%%%%%%%%%%%%%%%%%%%%%%%%%%%%%%%%%%%%%%%%%%%%%%
%%%%%%%%%%%%%%%%%%%%%%%%%%%%%%%%%%%%%%%%%%%%%%%%%%%%%%%%%%%%%%%%%%%%%%%%%%%%%%%%
\begin{document}

\title{边等周问题}
\maketitle

\section{基本定义}
\label{Section 1}

一个图(Graph)$G = (V, E, \partial)$ 由顶点集 $V$、边集 $E$
和标识每一条边对应一对顶点(可以相同)的边界函数
$\partial \colon E \rightarrow \binom{v}{1} \cup \binom{v}{2}$ 构成。
图经常用图表(Diagram)来表示,
其中顶点对应图表中的点,边对应图表中连接一对点的曲线。
对任意的图 $G$ 和 $S \subseteq V$,我们定义:
\begin{equation*}
\Theta(S) = \{e \in E \mid \partial(e) = \{v, w\}, v \in S, w \notin S\}
\end{equation*}
并称之为 $S$ 的边界(Edge-boundary)。
那么对于一个给定的图 $G$ 和 $k \in \mathbb{Z}^+$,
边等周问题(Edge-Isoperimetric Problem, EIP)是对于所有的 $S \subseteq V$ 并且 $|S| = k$,
找出最小的 $|\Theta(S)|$。
注意 $|\Theta(S)|$ 是不变量(Invariant),
即如果 $\phi \colon G \rightarrow H$ 表示一个图同构,
那么对于 $\forall S \subseteq V_G$,都有 $|\Theta(\phi(S))| = |\Theta(S)|$。
因此,在一个自同构下等价的顶点子集有相同的边界。

自环,即只对应一个顶点的边,与 EIP 无关,因此我们将忽略它们。
我们大多数的图(但并非全部)是简单图(Ordinary Graph),即不含自环和多重边。
一个简单图的表示可以缩短至 $(V, E)$,
其中 $E \subseteq (_2^v)$,$\partial$ 则是隐含的。

\section{例子}
\label{Section 2}

\subsection{$K_n$,$n$-完全图}
\label{Subsection 2.1}

$K_n$ 有 $n$ 个顶点且 $E = \binom{v}{2}$,即每一对不同顶点之间都有一条边。
对于每一个 $S \subseteq V$ 且 $|S| = k$,
$|\Theta(S)| = |S \times (V − S)| = k(n − k)$。
因此 $K_n$ 的 EIP 很简单:任意一个 $k$-集都是解。

\subsection{$\mathbb{Z}_n$,$n$-环}
\label{Subsection 2.2}

对于 $\mathbb{Z}_n$,有 $V = \{0, 1, \dots, n − 1\}$ 和
$E = \{\{i, j\} \mid i − j \equiv \pm 1 \pmod n\}$。因此,$\mathbb{Z}_3 = K_3$,
$\mathbb{Z}_4$ 的图表见图 \ref{Figure 1}。

\begin{figure}
	\centering
	\includesvg{figure-1}
	\caption{$\mathbb{Z}_4$ 的图表}
	\label{Figure 1}
\end{figure}

现在我们从下面的一般性说明(这些说明在后面也会有用),来推导 $\mathbb{Z}_4$ 以及
$\mathbb{Z}_n$ 上 EIP 的解:

\begin{enumerate}[(1)]
	\item \begin{enumerate}[(a)]
		\item 对于 $|S| = k = 0$,在任意的图里都只有一个子集,即空集 $\emptyset$。
			因此 $\min_{|S| = 0} |\Theta(S)| = |\Theta(\emptyset)| = 0$。
		\item 对于 $k = |V| = n$,也只有一个子集,即 $V$。
			因此 $\min_{|S| = n} |\Theta(S)| = |\Theta(V)| = 0$。
		\end{enumerate}
	\item 一个图被称为 $\delta$ 度正则图(Regular of Degree $\delta$),
		当它的每个顶点都正好对应 $\delta$ 条边。在一个正则图中,
		如果 $|S| = k = 1$ 那么 $\Theta(S) = \delta$,
		因此任意单点集都是一个解集。$\mathbb{Z}_n$ 是 $2$ 度正则图;
		然而对于 $n = 4$ 且 $k = 2$ 的情况,有两个集合并不在 $\mathbb{Z}_n$
		的对称性下等价:$\{0, 1\}$ 和 $\{0, 2\}$。
		所有其他的 $2$-集都与这两者中的一个等价。$|\Theta(\{0, 1\})| = 2$,
		$|\Theta(\{0, 2\})| = 4$,因此 $\min_{|S| = 2} |\Theta(S)| = 2$。
	\item 对于 $\forall G$ 和 $\forall S \subseteq V$,有
		\begin{equation*}
		\Theta(V − S) = \Theta(S)
		\end{equation*}
		因此对于 $k > \frac{1}{2} |V|$,
		$\min_{|S| = k} |\Theta(S)| = \min_{|S| = n − k} |\Theta(S)|$,
		其中 $n = |V|$。这样我们就完成了 $\mathbb{Z}_4$ 上 EIP 的解,
		见下表总结。
		\begin{center}
			\begin{tabular}{ c | c c c c c }
			$k$                          & 0 & 1 & 2 & 3 & 4 \\
			\hline
			$\min_{|S| = k} |\Theta(S)|$ & 0 & 2 & 2 & 2 & 0
			\end{tabular}
		\end{center}
	\item 令
		\begin{equation*}
		E(S) = \{e \in E \mid \partial(e) = \{v, w\}, v \in S, w \in S\}
		\end{equation*}
		$E(S)$ 被称为 $S$ 的导出边(Induced Edges)。
		对于一个图,导出边问题(Induced Edge Problem)是对于所有的
		$S \subseteq V$ 并且 $|S| = k$,找出最大的 $|E(S)|$。
\end{enumerate}

\begin{lemma}
\label{Lemma 1}
如果图 $G = (V, E, \partial)$ 是一个 $\delta$ 度正则图,
那么对于 $\forall S \subseteq V$,都有
\begin{equation*}
|\Theta(S)| + 2 |E(S)| = \delta |S|
\end{equation*}
\end{lemma}

\begin{proof}[证明]
$\delta |S|$ 表示 $S$ 对应的边数,然而出现在 $E(S)$ 中的边被计算了两次。
\end{proof}

\begin{corollary}
\label{Corollary 1}
如果 $G$ 是一个正则图,
那么 $S \subseteq V$ 是导出边问题的一个解当且仅当它同时也是 EIP 的解。
而且,对于 $\forall k$,$\min_{|S| = k} |\Theta(S)| = \delta k − 2 \max_{|S| = k} |E(S)|$。
\end{corollary}

那么对于正则图,EIP 和导出边问题是等价的,我们将视它们为可等价互换的问题。
通常 EIP 出现在实际应用中,而导出边问题则更易证明。EIP 还存在第三种自然变体:
对于 $S \subseteq V$,令
\begin{equation*}
\partial^\ast(S) = \{e \in E \mid \partial(e) \cap S \neq \emptyset\}
\end{equation*}
即 $S$ 中的点对应的边的集合。

\begin{exercise}
\label{Exercise 1}
对于正则图,证明计算
\begin{equation*}
\min_{\substack{
	S \subseteq V \\
	|S| = k
}}|\partial^\ast(S)|
\end{equation*}
与 EIP 等价 \footnote{练习的解答见第 \ref{Subsection 5.2} 小节,下同。}。
\end{exercise}

回想一下,一棵树(Tree)是连通无环图。一个无环图也被称为森林(Forest),因为它是
树——其连通分支的并集。

\begin{lemma}
\label{Lemma 2}
有 $n$ 个顶点的树的边数为 $n − 1$。有 $n$ 个顶点的森林的导出边数则是 $n - t$,
其中 $t$ 为其连通分支数。
\end{lemma}

$\mathbb{Z}_n$ 的任意适当的子集 $S$ 都可以导出一个无环图,
因此 $\max_{|S| = k} |\Theta(S)|$ 会出现于当 $S$ 是一个连通的集合,即区间时。
因而,如果 $0 < k < n$,$\min_{|S| = k} |\Theta(S)| = 2 k − 2 (k − 1) = 2$。

\subsection{$Q_d$,$d$ 阶立方}
\label{Subsection 2.3}

$d$ 阶立方 $Q_d$ 有顶点集 $\{0, 1\}^d$,即 $\{0, 1\}$ 的 $d$ 倍笛卡尔积。
因此 $n = |V_{Q_d}| = 2^d$。$Q_d$ 中的两个顶点($0$ 和 $1$ 构成的 $d$ 元组)对应一条边,
当且仅当它们正好有一位不相同。

\begin{exercise}
\label{Exercise 2}
找出 $m = |E_{Q_d}|$ 对应的公式。
\end{exercise}

$Q_1$ 和 $K_2$ 同构,$Q_2$ 和 $\mathbb{Z}_4$ 同构,这些都已经可以通过 EIP 得到解决。
一个 $3$ 阶立方有 $8$ 个顶点,$12$ 条边,以及 $6$ 个平面。
$Q_3$ 的图表事实上是一个 $3$ 阶立方的投影,见图 \ref{Figure 2}。

\begin{figure}
	\centering
	\includesvg{figure-2}
	\caption{$Q_3$ 的图表}
	\label{Figure 2}
\end{figure}

我们可以用在前两个例子中发展出来的简便工具来解决 $Q_3$ 上的 EIP。
首先观察到 $Q_3$ 有围长(最短回路的长度)为 $4$:
因为 $3$ 阶立方的对称群是传递的,所以任意顶点都和其他顶点一样。
从一个顶点出发勾勒出路径,我们可以看到不存在长度为 $3$ 的闭合回路。因此对于
$1 \le k \le 3$,根据引理 \ref{Lemma 1} 和引理 \ref{Lemma 2},我们有
\begin{align*}
\min_{|S| = k} |\Theta(S)| & = 3 k - 2 \max_{|S| = k} |E(S)| \\
			   & = 3 k - 2 (k - 1) = k + 2
\end{align*}
对于 $k = 4$,要么 $S$ 导出一个回路,在这种情况下是一个 $4$-环,并且 $|\Theta(S)| = 4$;
要么 $S$ 导出一个无环图,并且根据上述可以得到 $|\Theta(S)| \ge 6$。
对于 $k > 4 = \frac{8}{2}$,我们根据下面这个式子得到解
\begin{equation*}
\min_{|S| = k} |\Theta(S)| = \min_{|S| = 8 - k} |\Theta(S)|
\end{equation*}
最后的解见下表总结。
\begin{center}
	\begin{tabular}{ c | c c c c c c c c c }
	$k$                          & 0 & 1 & 2 & 3 & 4 & 5 & 6 & 7 & 8 \\
	\hline
	$\min_{|S| = k} |\Theta(S)|$ & 0 & 3 & 4 & 5 & 4 & 5 & 4 & 3 & 0
	\end{tabular}
\end{center}

为了将这个 EIP 的解推广到 $Q_d$($d > 3$),我们需要利用一些关于立方的简单结论,
这些结论的证明将被留作练习。
一个 $d$ 阶立方的 $c$ 阶子立方($c$-subcube of the $d$-cube)是 $Q_d$ 的子图,
导出自一个包含所有在 $(d − c)$ 个坐标下有相同(固定)值的顶点的集合。

\begin{exercise}
\label{Exercise 3}
证明 $d$ 阶立方的任意 $c$ 阶子立方与 $c$ 阶立方同构。
\end{exercise}

\begin{exercise}
\label{Exercise 4}
计算 $d$ 阶立方有多少个 $c$ 阶子立方?
\end{exercise}

一个 $d$ 阶立方的 $c$ 阶子立方的邻居(Neighbor)是在 $(d − c)$
个固定坐标下有正好一个坐标不同的任意 $c$ 阶子立方。

\begin{exercise}
\label{Exercise 5}
证明一个 $c$ 阶子立方的所有邻居都互不相交。
\end{exercise}

\begin{exercise}
\label{Exercise 6}
证明一个 $c$ 阶子立方的两个不同邻居(的顶点)之间没有边相连。
\end{exercise}

\begin{exercise}
\label{Exercise 7}
计算 $d$ 阶立方的一个 $c$ 阶子立方有多少个邻居?
\end{exercise}

$Q_d$ 上的 EIP 最初由数据传输上的问题推动。
W. H. Kautz \cite{Kautz.1954}、E. C. Posner 和本书作者的研究引发了一种猜测:
字典序编号方式(Lexicographic Numbering)的起始部分,即对于 $x \in V_{Q_d}$,
\begin{equation*}
lex(x) = 1 + \sum_{i = 1}^d x_i 2^{i - 1}
\end{equation*}
是解集,然而这要如何证明?一个显而易见的尝试方法是对维度 $d$ 使用归纳法。
数学归纳法有一个看似自相矛盾的属性,它往往更容易证明一个更强的定理,
因为一旦最初的情况得到验证,人们就可以假设该定理对归纳参数的较低值为真,
以便建立下一次归纳。因此一个较强的假设可以产生一个更简单的证明。
在本例中,该策略引发了这种推测——以下归纳步骤会生成 \emph{所有} 解集:

\begin{enumerate}[(1)]
	\item 首先从空集 $\emptyset$ 开始。
	\item 对于已经构建好的集合 $S \subset V_{Q_d}$,
		选择任意一个 $x \in V_{Q_d} − S$ 来扩充 $S$,使得导出边数的增量最大,即
		\begin{equation*}
		|E(S \cup \{x\})| − |E(S)|
		\end{equation*}
\end{enumerate}

添加任意 $x \in V_{Q_d}$ 都可使 $\emptyset$ 的增量最大,
因为 $|E(\{x\})| − |E(\emptyset)| = 0$。$\{x\}$ 的增量则必须是 $x$ 的相邻顶点。
那么对于 $k > 2$ 的 $k$-集,情况又该如何呢?答案是如果 $k = 2^c$,
那么该集合必然是 $c$ 阶子立方。我们刚刚已经验证了 $c = 0$ 和 $1$ 的情况。
假设这对 $0, 1, \dots, c − 1$ 都成立,为了扩充一个 $2^{c − 1}$-集,
即一个 $(c − 1)$ 阶子立方,我们只能选择一个使 $|E(S)|$ 增量为 $1 $的顶点,
即一个相邻的 $(c − 1)$ 阶子立方中的任意顶点。从一个相邻的子立方中选出一个顶点后,
我们必须继续从同一个子立方中选顶点,直到用尽它所有的顶点,
这是因为所选择的子立方中总会存在一个顶点使得 $|E(S \cup \{x\})| − |E(S)| \ge 2$,
而其他子立方中的任意顶点都有 $|E(S \cup \{x\})| − |E(S)| \le 2$。
当我们用尽所有相邻的 $(c − 1)$ 阶子立方时,我们就得到了一个 $c$ 阶子立方。

总体而言,令
\begin{equation*}
k = \sum_{i = 1}^K 2^{c_i}, 0 \le c_1 < c_2 < \cdots < c_K \footnotemark
\end{equation*}
\footnotetext{这里 $c_K \le \lfloor \log_2 k \rfloor$。}
为 $k$ 的二进制表示形式(注意 $K = \lfloor \log_2 k \rfloor$
\footnote{该处有错误,应为 $K \le \lceil \log_2 k \rceil$。})。
如果 $S \subseteq V_{Q_d}$ 是一个互不相交的子立方的并集,
其中每个 $c_i$ 阶子立方($1 \le i \le K$)都满足:对于其他所有 $c_i$ 阶子立方($j > i$),
该 $c_i$ 阶子立方处在该 $c_j$ 阶子立方的一个邻居中,那么 $S$ 就被称为立方集(Cubal Set)。
立方集正好是依次使内部边数增量最大化所构造出来的集合。如果 $S$ 是立方集且 $|S| = k$,那么
(见练习 \ref{Exercise 1})
\begin{equation*}
|E(S)| = \sum_{i = 1}^K (K - i) 2^{c_i} + c_i 2^{c_i - 1}
\end{equation*}
注意对于一个 $k$-立方集 $S \subseteq V_{Q_d}$,$|E(S)|$ 并非由 $d$ 决定,
而仅仅由 $k = |S|$ 决定。这个函数很重要,我们把它记为 $E(k)$。$E(k)$ 有一个分形性质,
见下面这个递推式:如果 $2^{d - 1} \le k < 2^d$,那么
\begin{equation*}
E(k + 1) − E(k) = E(k − 2^{d − 1} + 1) − E(k − 2^{d − 1}) + 1
\end{equation*}
这遵循 $k$-立方集的递归结构。从 $k$ 中减去最大的 $2$ 的幂——$2^{d − 1}$,
对应从 $S$ 中去掉最大的子立方。这个子立方为集合中每一个余下的顶点提供了一个邻居。

\begin{exercise}
\label{Exercise 8}
证明如果 $S \subseteq V_{Q_d}$ 是立方集,那么它的补集 $V_{Q_d} − S$ 也是立方集。
\end{exercise}

\begin{exercise}
\label{Exercise 9}
证明任意两个 $k$-立方集都同构,
即存在一个 $d$ 阶立方的自同构能把一个($k$-立方集)映射到另一个。
\end{exercise}

\begin{theorem}
\label{Theorem 1}
$S \subseteq V_{Q_d}$ 在基数为 $k$ 时有最大的 $|E(S)|$,当且仅当 $S$ 是立方集。
\end{theorem}

\begin{lemma}
\label{Lemma 3}
(Bernstein \cite{Bernstein.1967})
对于 $\forall d$ 和 $\forall k, t > 0$ 使得 $k + t < 2^d$,都有
\begin{equation*}
E(t) < E(k + t) − E(k) < E(2^d) − E(2^d − t)
\end{equation*}
\end{lemma}

\begin{proof}[引理 \ref{Lemma 3} 的证明]
对 $d$ 进行归纳:当 $d = 2$ 时成立;假设 $d − 1 \ge 2$ 时也成立,考虑下面三种情况:
\begin{enumerate}[(1)]
	\item \label{Lemma 3 Case 1}
		如果 $k \ge 2^{d − 1}$,那么有
		\begin{align*}
		E(k + t) − E(k)
		& = \sum_{i = 1}^t E(k + i) − E(k + i − 1) \\
		& = \sum_{i = 1}^t \left[
			E(k + i − 2^{d − 1}) − E(k + i − 2^{d − 1} − 1) + 1
		\right] \\
		& = E(k + t − 2^{d − 1}) − E(k − 2^{d − 1}) + t
		\end{align*}
		并且两个不等式都遵循归纳假设。
	\item \label{Lemma 3 Case 2}
		如果 $k + t \le 2^{d − 1}$,那么左边的不等式遵循归纳假设;
		右边的不等式遵循上述恒等式,因而也遵循归纳假设。
	\item \label{Lemma 3 Case 3}
		如果 $k < 2^{d − 1} < k + t$,那么
		\begin{align*}
		E(k + t) − E(k) & = \left[E(k + t) − E(2^{d − 1})\right] +
				    \left[E(2^{d − 1}) − E(k)\right] \\
				& > \left[E(k + t − 2^{d − 1})\right] +
				    \left[E(t) − E(t − (2^{d − 1} − k))\right] \\
		\intertext{\hfill 分别根据情况 \ref{Lemma 3 Case 1} 和 \ref{Lemma 3 Case 2}
		\footnotemark}
				& = E(t)
		\end{align*}
		\footnotetext{该处证明不完整,详见第 \ref{Subsection 5.1} 节。}
\end{enumerate}

\begin{exercise}
\label{Exercise 10}
完成情况 \ref{Lemma 3 Case 3} 的证明(右边的不等式)。
\end{exercise}
\end{proof}

\begin{proof}[定理 \ref{Theorem 1} 的证明]
我们已经注意到所有的 $k$-立方集都有相同的导出边数,即 $E(k)$,
因此我们只需证明所有的最优集都是立方集。先前我们对 $d$ 进行归纳,
并已证明了当 $d = 1, 2$ 时为真。现在假设当维度 $d > 2$ 时也为真。
对于给定的 $k$($0 < k < n = 2^d$)并用上述 $2$ 的幂的和来表示(所以 $K < d$),
令 $S \subseteq V_{Q_d}$ 为 $|S| = k$ 时的最优集。如果我们根据第 $d$ 位坐标,
把 $Q_d$ 划分为两个 $(d − 1)$ 阶子立方
$Q_{d, 0} = \{x \in V_{Q_d} \mid x_d = 0\}$ 和
$Q_{d, 1} = \{x \in V_{Q_d} \mid x_d = 1\}$,
那么我们也得到了 $S$ 的划分,即
$S_0 = S \cap Q_{d, 0}$ 和 $S_1 = S \cap Q_{d, 1}$。
令 $|S_0| = k_0$,$|S_1| = k_1$,我们不妨假设 $k_0 \ge k_1$。
如果 $k_1 = 0$,定理遵循归纳假设,因此就假设 $k_1 > 0$。
因为 $E(S)$ 中的边的端点要么都在 $S_0$ 中,要么都在 $S_1$ 中,
要么一个在 $S_0$ 中,一个在 $S_1$ 中,所以
\begin{equation*}
|E(S)| \le \max_{|S| = k_0} |E(S)| + \max_{|S| = k_1} |E(S)| + k_1
\end{equation*}
如果 $S_1$ 是基数为 $k_1$ 的立方集,那么根据归纳,我们有
$|E(S_1)| = E(k_1) = \max_{|S| = k_1} |E(S)|$。
$S_1$ 在 $Q_{d, 0}$ 中的邻居与 $S_1$ 同构,因此也是立方集。
根据立方集的递归构造,$\exists S_0 \subseteq Q_{d, 0}$ 且 $|S_0| = k_0$,
它是立方集,并且包含 $S_1$ 的邻居。因此我们可以取到 $|E(S)|$ 的上界,
而且每个取到上界的集合都必须是两个立方集的并集。
令 $k_0 = \sum_{i = 1}^{K_0} 2^{c_{i, 0}}, 0 \le c_{1, 0} \le c_{2, 0} \le \cdots \le c_{K_0, 0}$,
$k_1$ 与其类似。因为 $k_0 + k_1 = k$,所以只有三种可能的情况:
\begin{enumerate}[(1)]
	\item $c_{K_0, 0} = c_K$:
		那么我们可以假设 $S_0 - Q_{c_{K_0, 0}}$ 和 $S_1$ 处在
		$Q_{c_{K_0, 0}}$ 的两个不同的邻居中,因此 $S$ 不是立方集。
		根据练习 \ref{Exercise 5} 可知 $S_0 - Q_{c_{K_0, 0}}$
		中的顶点和 $S_1$ 中的顶点没有边相连。
		令 $k_0^{'} = k_0 - 2^{c_{K_0}} > 0$,
		我们有 $k_0^{'} + k_1 \le 2^{c_{K_0}}$。
		如果我们去掉 $S_1$ 并添加相同数量的顶点至 $S_0$,以此来修改 $S$,
		那么引理 \ref{Lemma 3} 表明 $|E(S)|$ 会增大。
		这便与 $S$ 是最优的相矛盾。
	\item $c_{K_0, 0} = c_K - 1$ 且 $c_{K_1, 1} = c_K - 1$:
		$Q_{c_{K_0, 0}}$ 和 $Q_{c_{K_1, 1}}$,
		这两个 $(c_K − 1)$ 阶子立方是邻居,因此构成了一个 $c_K$ 阶子立方。
		$S_0 - Q_{c_{K_0, 0}}$ 和 $S_1 - Q_{c_{K_1, 1}}$
		每一个都处在相邻的 $(c_K − 1)$ 阶子立方中,
		这两个子立方构成了一个与第一个相邻的 $c_K$ 阶子立方。
		根据归纳假设,$S$ 肯定是立方集。
	\item $c_{K_0, 0} = c_K - 1$ 且 $c_{K_1, 1} < c_K - 1$:
		与情况 1 同理,我们假设 $S_0 - Q_{c_{K_0, 0}}$ 和 $S_1$ 处在
		$Q_{c_{K_0, 0}}$ 的两个不同的邻居中,因此它们之间没有边相连。
		令 $k_0^{'} = k_0 - 2^{c_{K_0}} > 0$,我们有
		\begin{align*}
		k_0^{'} + k_1 & = k_0 - 2^{c_{K_0}} + k_1 \\
			      & = k - 2^{c_K - 1} \\
			      & \ge 2^{c_K} - 2^{c_K - 1} \\
			      & = 2^{c_K - 1}
		\end{align*}
		如果我们从 $S_1$ 中去掉 $2^{c_{K_0}} − k_0^{'}$ 个顶点
		并把它们补充到包含 $S_0 - Q_{c_{K_0, 0}}$ 的那个集合中,
		使其成为 $Q_{c_{K_0, 0}}$ 的邻居,以此来修改 $S$,
		那么引理 \ref{Lemma 3} 表明这会使 $|E(S)|$ 增大,
		再一次与 $S$ 是最优的相矛盾。
\end{enumerate}
\end{proof}

\section{布局问题上的应用}
\label{Section 3}

组合等周问题频繁地出现在通信工程、计算机科学、物理科学和数学本身这些领域。
我们不希望在这里涵盖所有的应用,但会给出一个具有代表性的样例。
我们选择专注于布局问题,它们出现在电气工程中,
当一个人把一些电气回路的布线图在机箱上“勾画出来”,即把每一个元件和线放置在机箱上。
布线图本质上就是一个图,其中电气元件成了顶点,连接它们的线成了边。
对于机箱上的顶点和边的任意给定布局,都存在一定的成本或者说性能度量,我们则希望对其进行优化。

\subsection{线长问题}
\label{Subsection 3.1}

假设我们想把元件(图 $G = (V, E, \partial)$ 中的顶点)放置在一个线形的机箱上,
每一个都与先前的那个元件相距一个单位,以这样的方式来最小化连接它们的线的总长度。
精确地来说,我们将 $G$ 的一种顶点编号方式(Vertex-numbering)定义为一个一对一的函数
\begin{equation*}
\eta \colon E \rightarrow \{1, 2, \dots, n\},\ \text{其中}\ n = |V|
\end{equation*}
$\eta$ 的值域中的整数可根据线形机箱上的位置来区分。
那么 $\eta$ 的总线长(Wirelength)则是
\begin{equation*}
wl(\eta) = \sum_{\substack{
	e \in E \\
	\partial(e) = \{v, w\}
}} |\eta(v) - \eta(w)|
\end{equation*}

对于一个图 $G = (V, E, \partial)$ 来说,
\begin{equation*}
wl(G) = \min \{wl(\eta) \mid \eta\ \text{是}\ G\ \text{的一种顶点编号方式}\}
\end{equation*}
记住,一个有 $n$ 个顶点的图有 $n!$ 种顶点编号方式。

\subsubsection{例子}
\label{Subsubsection 3.1.1}

平方的图有 $4! = 24$ 种顶点编号方式,但它同时还有 $8$ 种对称情况。
任意两种对称的编号方式都有相同的线长。
图 \ref{Figure 3} 的三种编号方式代表了 $24 / 8 = 3$ 种编号方式的等价类。
前两种编号方式有最小的线长 $wl$,第三种则有最大的。因此 $wl(Q_2) = 6$。
然而 $wl(Q_3)$ 就没有那么容易得出了,
因为 $Q_3$ 有 $8! = 40320$ 种编号方式以及 $48$ 种对称情况。
虽然 $40320 / 48 = 840$ 也并不是那么大,
但是要如何才能系统地生成这 $840$ 种编号方式等价类的代表呢?
我们现在来看看如何解决这些明显的困难。

\begin{figure}
	\centering
	\includesvg[width = 1.0\textwidth]{figure-3}
	\caption{$Q_2$ 的编号方式}
	\label{Figure 3}
\end{figure}

为了最小化一个和,譬如 $wl$,一个显而易见的策略是分别最小化每一个加数。
这些最小值的和就成了和的最小值的下界,并且我们可以期望这会是一个不错的下界,甚至是精确的。
然而这并不适用于 $wl(\eta)$ 的定义,因为对于每一条边 $e \in E $,$\partial(e) = \{v, w\}$,都有
\begin{equation*}
\min_{\eta} |\eta(v) - \eta(w)| = 1
\end{equation*}

\subsubsection{$wl$ 的另一种表示形式}
\label{Subsubsection 3.1.2}

对于给定的一个编号方式 $\eta$ 和一个整数 $k$,$0 \le k \le n$,令
\begin{equation*}
S_k(\eta) = \eta^{-1}(\{1, 2, \dots, k\}) = \{v \in V \mid \eta(v) \le k\}
\end{equation*}
即编号方式 $\eta$ 下的前 $k$ 个顶点的集合。
那么下面我们就得到了线长的另一种表示形式。

\begin{lemma}
\label{Lemma 4}
\begin{equation*}
wl(\eta) = \sum_{k = 0}^n |\Theta(S_k(\eta))|
\end{equation*}
\end{lemma}

\begin{proof}[证明]
注意 $S_0(\eta) = \eta^{−1}(\emptyset) = \emptyset$。令
\begin{equation*}
\chi(e, k) = \begin{cases}
	1 & \text{如果}\ \partial(e) = \{v, w\}, \eta(v) \le k < \eta(w) \\
	0 & \text{其他情况}
\end{cases}
\end{equation*}
那么
\begin{align*}
wl(\eta)
& = \sum_{\substack{
	e \in E \\
	\partial(e) = \{v, w\}
}} |\eta(v) - \eta(w)| = \sum_{e \in E} \sum_{k = 0}^n \chi(e, k) \\
& = \sum_{k = 0}^n \sum_{e \in E} \chi(e, k) = \sum_{k = 0}^n |\Theta(S_k(\eta))|
\end{align*}
\end{proof}

\begin{exercise}
\label{Exercise 11}
(Steiglitz–Bernstein \cite{Steiglitz.1965})
假设线形机箱实际上是一条实线,上面有 $n$ 个位置 $s_1 < s_2 < \cdots < s_n$ 用来放置元件。
布局 $\eta$ 将顶点 $v \in V$ 放置在位置 $s_{\eta(v)}$ 上,证明 $\eta$ 有总线长为
\begin{equation*}
wl(\eta) = \sum_{k = 0}^n (s_{k + 1} - s_k) |\Theta(S_k(\eta))|
\end{equation*}
\end{exercise}

线长的这种作为一个和的新表示形式给了我们它的另一个下界。

\begin{corollary}
\label{Corollary 2}
对于任意的图 $G$,
\begin{align*}
wl(G)
& = \min_{\eta} wl(\eta) = \min_{\eta} \sum_{k = 0}^n |\Theta(S_k(\eta))| \\
& \ge \sum_{k = 0}^n \min_{\eta} |\Theta(S_k(\eta))| = \sum_{k = 0}^n \min_{|S| = k} |\Theta(S)|
\end{align*}
\end{corollary}

\begin{theorem}
\label{Theorem 2}
对于 $G = (V, E, \partial)$ 的任意顶点编号方式 $\eta_0$,
如果它的所有初始段 $S_k(\eta_0)$($0 \le k \le n$)都是 $G$ 上 EIP 的解,
那么它本身就是 $G$ 上线长问题的一个解。
\end{theorem}

\begin{corollary}
\label{Corollary 3}
$\mathbb{Z}_n$ 的编号方式 $\eta_0(i) = 1 + i$ 是 $\mathbb{Z}_n$ 上线长问题的一个解,并且
\begin{align*}
wl(\mathbb{Z}_n) & = \sum_{k = 0}^n \min_{|S| = k} |\Theta(S)| \\
		 & = \sum_{k = 1}^{n - 1} 2 = 2 (n - 1)
\end{align*}
\end{corollary}

\begin{definition}
\label{Definition 1}
如果 $\{T_i\}_{i = 1}^n$ 是一个完全有序集合的序列,
那么它们乘积的字典序 $T_1 \times T_2 \times \cdots \times T_n$ 是一个由 $x < y$ 定义的全序,
$x < y$ 成立当 $\exists m$ 使得 $x_1 = y_1$,$x_2 = y_2$,……,
$x_{m - 1} = y_{m - 1}$ 且 $x_m < y_m$。
一个集合的任意全序都对应一个编号方式,其中集合的第一个(最小的)元素编号为 $1$,
第二个编号为 $2$,以此类推。
$0$ 和 $1$ 构成的 $d$ 元组($Q_d$ 的顶点)上的字典序对应的编号方式是
$lex(x) = 1 + \sum_{i = 1}^d x_i 2^{i - 1}$
\end{definition}

\begin{corollary}
\label{Corollary 4}
$Q_d$ 的字典序编号方式是 $Q_d$ 上线长问题的一个解。
\end{corollary}

\begin{exercise}
\label{Exercise 12}
证明 $wl(Q_d) = 2^{d - 1} (2^d - 1)$。
\end{exercise}

\begin{exercise}
\label{Exercise 13}
对于练习 \ref{Exercise 11} 中的更一般的线形机箱,
证明 $lex$ 同时也是它的 $Q_d$ 上线长问题的一个解。
\end{exercise}

\subsection{$4$ 阶德布鲁因图}
\label{Subsection 3.2}

德布鲁因图是一个有向图(Directed Graph, Digraph),即存在两个边界函数
$\partial_\pm \colon E \rightarrow V$ 分别标识每一条边的头和尾。
有向图的图表中的每条边一般都有一个箭头指向它的头。
$d$ 阶德布鲁因图(the deBruijn Graph of Order $d$)$DB_d$ 有与 $Q_d$ 相同的顶点集
($0$ 和 $1$ 构成的所有 $d$ 元组的集合),而它的边集则完全不同。
$E_{DB_d}$是 $0$ 和 $1$ 构成的所有 $(d + 1)$ 元组的集合,并且
\begin{align*}
\partial_{-}(x_1, x_2, \dots, x_d, x_{d + 1}) & = (x_1, x_2, \dots, x_d) \\
\partial_{+}(x_1, x_2, \dots, x_d, x_{d + 1}) & = (x_2, \dots, x_d, x_{d + 1})
\end{align*}
关于德布鲁因图的更多信息,请参考 \cite{Golomb.1982}。
$DB_4$ 的图表见图 \ref{Figure 4}。

\begin{figure}
	\centering
	\includesvg{figure-4}
	\caption{$4$ 阶德布鲁因图}
	\label{Figure 4}
\end{figure}

对于一个编号方式 $\eta$,它的作为边界的和的线长表示形式,
及其在求解 $\mathbb{Z}_n$ 和 $Q_d$ 上的线长问题上的应用,
表明了一种可最小化任意图的线长的启发式方法:
给顶点编号为 $1, 2, \dots, k - 1, k, \dots, n$,从而最小化
\begin{equation*}
|\Theta(S_k(\eta))| - |\Theta(S_{k - 1}(\eta))|
\end{equation*}
即每增加一个顶点所带来的边际增量。
把这个启发式方法应用到 $\underline{DB_4}$(去掉自环和边的方向的 $DB_4$)上,
我们就得到了编号方式 $\eta_0$,见图 \ref{Figure 5}。

\begin{figure}
	\centering
	\includesvg{figure-5}
	\caption{编号方式 $\eta_0$ 下 $\underline{DB_4}$ 的图表}
	\label{Figure 5}
\end{figure}

下表列出了 $k$ 的取值范围在 $0$ 到 $16$ 时 $|\Theta(S_k(\eta_0))|$ 的值:
\begin{center}
	\begin{tabular}{ c | c c c c c c c c c c c c c c c c c }
	$k$                     & 0 & 1 & 2 & 3 & 4 & 5 & 6 & 7 & 8 & 9
	                        & 10 & 11 & 12 & 13 & 14 & 15 & 16 \\
	\hline
	$|\Theta(S_k(\eta_0))|$ & 0 & 2 & 4 & 4 & 6 & 6 & 6 & 6 & 6 & 6
	                        &  8 &  6 &  6 &  4 &  4 &  2 &  0
	\end{tabular}
\end{center}
从这可以看出 $S_{10}(\eta_0)$ 不是 EIP 的解。
事实上并不存在一个所有的初始段都是 EIP 的解的编号方式,
因为如果存在的话,它就会产生于我们用来构造 $\eta_0$ 的这个过程中。
然而,我们可以断言,对于 $k \neq 10$,$S_k(\eta_0)$ 是 EIP 的解。
因此有不等式可导致定理 \ref{Theorem 2},即
\begin{equation*}
wl(\underline{DB_4}) > \sum_{k = 0}^n \min_{|S| = k} |\Theta(S)| = 2 \times 34 + 6 = 74
\end{equation*}
而
\begin{equation*}
wl(\eta_0) = 76
\end{equation*}
因为对于每一个顶点 $v \in V_{\underline{DB_4}}$,度数 $\delta(v)$ 都是偶数,
并且对于所有的 $S \subseteq V_{\underline{DB_4}}$,
\begin{equation*}
|\Theta(S)| = \sum_{v \in S} \delta(v) - 2 |E(S)|
\end{equation*}
也是偶数(引理 \ref{Lemma 1} 中的恒等式的泛化),
所以对于任意 $\underline{DB_4}$ 的编号方式 $\eta$,我们都有 $wl(\eta) \ge 76$,
因此 $\eta_0$ 是它的线长问题的一个解。

\begin{exercise}
\label{Exercise 14}
找出 $\underline{DB_4}$ 上所有的 EIP 的解集,其中 $k \le 8$(无需证明)。
\end{exercise}

\subsection{划分问题}
\label{Subsection 3.3}

$G = (V, E, \partial)$ 的一个划分(Partition)是一个集合 $\pi \subseteq 2^V$ 满足
\begin{enumerate}[(1)]
	\item $\forall B \in \pi$,$B \neq \emptyset$;
	\item $\forall A, B \in \pi$,要么 $A = B$,要么 $A \cap B = \emptyset$;
	\item $\bigcup_{B \in \pi} B = V$。
\end{enumerate}

$B \in \pi$ 被称为划分的一个块(Block)。
一个划分 $\pi$ 被认为是均匀的(Uniform),当 $\forall A, B \in \pi$,$||A| − |B|| \le 1$。
如果 $|\pi| = p$,那么这就等价于
\begin{equation*}
\forall B \in \pi,
\left\lfloor\frac{n}{p}\right\rfloor \le |B| \le \left\lceil\frac{n}{p}\right\rceil,
\ \text{其中}\ n = |V|
\end{equation*}
事实上,如果 $j$ 是 $n$ 除以 $p$ 的余数(即 $n \equiv j \pmod p$,$0 \le j < p$),
那么 $G$ 的任意均匀 $p$-划分都有 $j$ 个 $\lceil\frac{n}{p}\rceil$-块
和 $(p - j)$ 个 $\lfloor\frac{n}{p}\rfloor$-块。

一个划分 $\pi$ 的边界(Edge-boundary)是
\begin{align*}
\Theta(\pi) & = \{e \in E \mid \partial(e) = \{v, w\}, v \in A, w \in B, A \ne B\} \\
	    & = \bigcup_{B \in \pi} \Theta(B)
\end{align*}
注意后一种表示形式中的并集不是互不相交的,
而是任意 $e \in E$ 且 $\partial(e) = \{v, w\}$ 都正好被包含了两次,
一次来自包含 $v$ 的块,另一次来自包含 $w$ 的块。

$G$ 的边界划分问题(Edge-boundary Partition Problem)是对于 $G$ 的所有均匀 $p$-划分 $\pi$,
找出最小的 $|\Theta(\pi)|$。$G$ 可以被当做一个要布局在 $p$ 块芯片上的布线图,
其中的元件要尽可能平均地分布在这些芯片上。问题则是给芯片分配元件,
使得连接不同芯片上元件的线数最小。
边界划分问题的一个变体是边宽划分问题(Edge-width Partition Problem),
即对于 $G$ 的所有均匀 $p$-划分 $\pi$,找出最小的 $\max_{B \in \pi} |\Theta(B)|$。

\begin{lemma}
\label{Lemma 5}
如果 $n \equiv j \pmod p$,$0 \le j < p$,那么
\begin{equation*}
\min_{\substack{
	|\pi| = p \\
	\pi \text{ 均匀}
}}|\Theta(\pi)| \ge \frac{1}{2} \left(
	j \min_{|B| = \lceil\frac{n}{p}\rceil} |\Theta(B)| +
	(p - j) \min_{|B| = \lfloor\frac{n}{p}\rfloor} |\Theta(B)|
\right)
\end{equation*}
同样还有
\begin{equation*}
\min_{|\pi| = p} \max_{B \in \pi} |\Theta(B)| \ge \max \left\{
	\min_{|B| = \lceil\frac{n}{p}\rceil} |\Theta(B)|,
	\min_{|B| = \lfloor\frac{n}{p}\rfloor} |\Theta(B)|
\right\}
\end{equation*}
\end{lemma}

\begin{proof}[证明]
如果 $\pi_0$ 是一个最优均匀 $p$-划分,那么
\begin{align*}
\min_{|\pi| = p} |\Theta(\pi)|
& = |\Theta(\pi_0)| \\
& = \frac{1}{2} \sum_{B \in \pi_0} |\Theta(B)| \\
\intertext{\hfill 因为每一条边 $e \in \Theta(\pi_0)$ 都被计算了两次}
& \ge \frac{1}{2} \left(
	j \min_{|B| = \lceil\frac{n}{p}\rceil} |\Theta(B)| +
	(p - j) \min_{|B| = \lfloor\frac{n}{p}\rfloor} |\Theta(B)|
\right)
\end{align*}
\end{proof}

\subsubsection{例子}
\label{Subsubsection 3.3.1}

\begin{enumerate}[(1)]
	\item 对于 $\mathbb{Z}_n$,我们得到
		\begin{align*}
		\min_{|\pi| = p} |\Theta(\pi)|
		& \ge \frac{1}{2} \left(
			j \min_{|B| = \lceil\frac{n}{p}\rceil} |\Theta(B)| +
			(p - j) \min_{|B| = \lfloor\frac{n}{p}\rfloor} |\Theta(B)|
		\right) \\
		& = \frac{1}{2} \left(
			j \cdot 2 + (p - j) \cdot 2
		\right) = p
		\end{align*}
		同样还有
		\begin{align*}
		\min_{|\pi| = p} \max_{B \in \pi} |\Theta(B)|
		& \ge \max\left\{
			\min_{|B| = \lceil\frac{n}{p}\rceil} |\Theta(B)|,
			\min_{|B| = \lfloor\frac{n}{p}\rfloor} |\Theta(B)|
		\right\} \\
		& = \max\{2, 2\} = 2
		\end{align*}
		这些下界可以在把 $\mathbb{Z}_n$ 均匀地划分为 $p$ 个区间时取到,
		因此 $\mathbb{Z}_n$ 上的边界和边宽划分问题都得到了解决。
	\item 对于 $Q_d$,$p = 2^a$ 是 $2$ 的幂,
		$\lceil\frac{n}{p}\rceil = \frac{2^d}{2^a} = 2^{d - a} = \lfloor\frac{n}{p}\rfloor$,
		因此我们得到
		\begin{align*}
		\min_{|\pi| = p} |\Theta(\pi)|
		& \ge \frac{1}{2} \left(
			j \min_{|B| = \lceil\frac{n}{p}\rceil} |\Theta(B)| +
			(p - j) \min_{|B| = \lfloor\frac{n}{p}\rfloor} |\Theta(B)|
		\right) \\
		& = \frac{1}{2} \cdot 2^a \min_{|B| = 2^{d - a}}|\Theta(B)|
		  = \frac{1}{2} \cdot 2^a \cdot (d - a) 2^{d - a} \\
		& = (d - a) 2^{d - 1}
		\end{align*}
		同样不出意料地,这个下界,以及与边宽对应的那个下界,
		可以在把 $Q_d$ 划分为 $(d - a)$ 阶子立方时取到。
		不过令人惊讶是,根据 Bezrukov \cite{Bezrukov.1997} 的观察,
		对于 $p$ 的其他(也许是全部)的值,这些下界依然是精确的。
\end{enumerate}

\begin{theorem}
\label{Theorem 3}
$\forall d > a$,都存在一个均匀 $(2^a + 1)$-划分,能把 $Q_d$ 划分为立方集。
\end{theorem}

\begin{exercise}
\label{Exercise 15}
在阅读下面的证明之前,先证明 $a = 1$ 时的特例。
\end{exercise}

\begin{proof}[证明]
我们从 $Q_d$ 中的一个 $\lfloor\frac{2^d}{2^a + 1}\rfloor$-立方集 $B_1$ 开始。
因为所有的 $k$-立方集都在图同构下等价(见练习 \ref{Exercise 9}),
我们不妨令 $B_1 = S_k(lex)$,其中 $k = \lfloor\frac{2^d}{2^a + 1}\rfloor$。
因为 $\lfloor\frac{2^d}{2^a + 1}\rfloor < 2^{d - a}$,
所以 $B_1$ 是 $Q_d$ 的 $(d - a)$ 阶子立方的子集,它的最后 $a$ 位坐标均为固定值 $0$。
这些固定坐标的 $2^a$ 个值每一个都给出 $B_1$ 的一份拷贝,因此都是立方集。
以这些固定坐标的字典序把它们标号为 $B_1, B_2, \dots, B_{2^a}$。
再把元素 $S_{k + 1}(lex) - S_k(lex)$ 加入到 $B_1$,
把对应的元素加入到 $B_2, \dots, B_j$,其中 $j \equiv 2^d \pmod{2^a + 1}$。
那么这些前 $j$ 个块就成了 $S_{k + 1}(lex)$ 的拷贝。我们断言
\begin{equation*}
\bigcup_{i = 1}^{2^a} B_i
\end{equation*}
是立方集:它包括一个 $(a_i + a)$ 阶子立方的并集,其中每一个的指数满足 $\sum 2^{a_i} = k$
并且每一个 $(a_i + a)$ 阶子立方都处在所有更大的子立方的邻居中。
它还包括一个 $b_l$ 阶子立方的并集,
其中每一个的指数满足 $\sum 2^{b_l} = j$,$b_1 < b_2 < \cdots$。
因为所有的 $b_l$ 阶子立方都处在一个 $a$ 阶子立方中,
它处在所有的 $(a_i + a)$ 阶子立方的一个邻居中,所以这个断言得到了证明。
因此,根据练习 \ref{Exercise 8},$B_{2^a + 1} = V_{Q_d} - \bigcup_{i = 1}^{2^a} B_i$
是一个大小为 $\lfloor\frac{2^d}{2^a + 1}\rfloor$ 的立方集,从而
\begin{equation*}
\pi = \{B_1, B_2, \dots, B_{2^a}, B_{2^a + 1}\}
\end{equation*}
是所求的划分。
\end{proof}

\begin{lemma}
\label{Lemma 6}
如果存在一个均匀 $p$-划分能把 $Q_d$ 划分为立方集,
那么也存在一个均匀 $2 p$-划分能把 $Q_{d + 1}$ 划分为立方集。
\end{lemma}

\begin{proof}[证明]
留作练习。
\end{proof}

Bezrukov \cite{Bezrukov.1997} 继续用类似的方法证明了对于固定的 $p$,
当 $d \rightarrow \infty$ 时,存在一个均匀 $p$ 划分能把 $Q_d$ 划分为渐进最优集。

\section{评论}
\label{Section 4}

\begin{displayquote}[Robert Browning]
一个人总是要不断超越自我,否则还要天堂做什么?
\end{displayquote}

当我在写给本章奠定基础的论文 \cite{Harper.1964} 的时候,我还只是一个初期的研究生。
当时的我毫无疑问超越了自我,甚至也许足以让 Browning 也赞同这个说法。
很幸运,我并不需要等到上了天堂才有机会弥补我的不足。
在我的论文发表两年后,Bernstein 发表了一篇后续论文 \cite{Bernstein.1967},
指出我忽视了一种情况(定理 \ref{Theorem 1} 论据中的情况 \ref{Lemma 3 Case 3})
并填补了这个空缺。当时 Bernstein 的补丁(引理 \ref{Lemma 3})看似复杂得令人失望,
现在,有了将近四十年该教材的使用经验,我了解到 Bernstein 的引理正是当时所需的。
它不仅涵盖了遗漏的情况 \ref{Lemma 3 Case 3},
还将情况 \ref{Lemma 3 Case 1} 中含糊不清的论据替换为一个清晰明确的说法,
同时还包含着更深刻的见解。

寻找最优均匀 $2$-划分的问题是 EIP 的一个很吸引人的特例,
它在文献中有许多不同的名称,比如“图二分”或者“最小平衡割”。

我们所谓的线长问题出现在许多不同的应用中,因此在文献中也有不同的名称。
我们之所以称之为“线长”是因为它既简洁又具有描述性。
$\mathbb{Z}_n$ 上线长问题的解由 Lehman \cite{Lehman.1963} 发表于 1963 年。
它的应用是显示如何用 $n$ 个不同重量但由均匀的弹簧相连的珠子来构造一个手镯,
以便最小化基本频率。$Q_d$ 上线长问题的解的原始应用 \cite{Harper.1964}
是将数据从集合 $\{1, 2, \dots, 2^d\}$ 编码到 $0$ 和 $1$ 的 $d$ 元组,
以便在一个有噪声(二元对称)但任意位发生错误概率都较低的信道上传输时最小化平均绝对误差。
这类线长问题的原始实例是 $4$ 阶德布鲁因图 \cite{Harper.1970},
它出现在解码电路的布线图中。最小化线长意味着最小化自感。

$Q_d$ 上的线长问题,即在本章出现的一个应用,实际上曾是我的工作的出发点。
我在喷气推进实验室(JPL)的老板 Ed Posner 向我提出了这个问题。
然而他所想的应用并不是最小化线长,
而是最小化在二元对称的信道上传输线性数据时的平均绝对误差。
当时 JPL 的摄像机正在传输第一份月球表面的近拍图像。
图像的像素是灰度,有 $64$ 级,从白到黑。它们被编码为 $0$ 和 $1$ 的 $6$ 元组,
用以传输到地球,然后再被解码为灰度并重组为图像。
问题是由于发送器仅被最低限度供电,在传输过程中一个 $0$ 会偶尔变成一个 $1$,或者相反。
这最终就会导致一个错误的灰度(取决于编码)并且恶化收到的图像。
当时面临的挑战是要证明工程师使用的编码,即字典序编码(编号方式),能最小化平均绝对误差。

作为一名本科生,在解决问题上我深深折服于 G. Polya 的著作 \cite{Polya.1954}。
Polya 的论题是解决问题是可以学习的,尤其是存在有效的策略可以有意识地施加于上。
其中之一便是将一个猜想 A 归约为——假定是更简单的——猜想 B 和 C 的结合。
我在致力于解决 Posner 的问题上遵循了 Polya 的建议,
这就导致产生了第 \ref{Subsection 3.1} 小节里的思路。
当我看到 Posner 的猜测可归约为字典序编号方式的初始段解决了 EIP 这一猜想,
并且这个猜想似乎适用于对 $d$ 进行归纳时,我很自信我有了重大发现。
自从认识到 Polya 的方法对我自己如此有帮助,我就一直在向年轻的数学家们推荐它。

能够给出 $Q_d$ 上 EIP 的所有解集是一个偶然的错误。
在那时我以为这对于归纳证明的逻辑来说是必要的。
现在回想起来,情况显然并非如此,
但它确实导致了一个更强有力的结果,使得这个结果能被更加灵活地应用起来。
注意 Bezrukov 在边界划分问题上的应用就利用了这种灵活性,
然而这是发生在这篇文章 \cite{Harper.1964} 发表的 33 年后了。

\section{译者后记}
\label{Section 5}

\subsection{错误与指正}
\label{Subsection 5.1}

在仔细阅读引理 \ref{Lemma 3} 的证明以及该证明的来源——Bernstein 的论文
\cite{Bernstein.1967} 后,我(译者,下同)惊奇地发现虽然 Bernstein 为 Harper
的定理 \ref{Theorem 1} 打上了补丁,然而该补丁依然有纰漏之处:
该证明的第 \ref{Lemma 3 Case 3} 部分并未考虑 $t \ge 2^{d - 1}$ 的情况,
即当 $k < 2^{d − 1} < k + t$ 时,
\begin{equation*}
E(2^{d − 1}) − E(k) > E(t) − E(t − (2^{d − 1} − k))
\end{equation*}
仅对 $t < 2^{d - 1}$ 成立。下面的补充证明由我的导师给出。
\begin{proof}[引理 \ref{Lemma 3} 情况 \ref{Lemma 3 Case 3} 的补充证明]
如果 $k < 2^{d − 1} < k + t$ 且 $t \ge 2^{d - 1}$,那么
\begin{align*}
E(k + t) - E(k) & = \left[E(k + t) - E(2^{d - 1})\right] +
		    \left[E(2^{d - 1}) - E(k)\right] \\
		& = \left[E(k + t - 2^{d - 1}) + k + t - 2^{d - 1}\right] +
		    \left[E(2^{d - 1}) - E(k)\right] \\
		& = \left[E(k + t - 2^{d - 1}) - E(k)\right] +
		    E(2^{d - 1}) + k + t - 2^{d - 1} \\
		& > E(t - 2^{d - 1}) + E(2^{d - 1}) + t - 2^{d - 1} + k \\
		& = E(t) + k > E(t)
\end{align*}
\end{proof}

\subsection{练习解答}
\label{Subsection 5.2}

\begin{exercisewithanswer}
对于正则图,证明计算
\begin{equation*}
\min_{\substack{
	S \subseteq V \\
	|S| = k
}}|\partial^\ast(S)|
\end{equation*}
与 EIP 等价。
\end{exercisewithanswer}

\begin{proof}[证明]
根据 $\partial^\ast(S)$ 的定义,我们有
\begin{equation*}
|\partial^\ast(S)| = |\Theta(S)| + |E(S)|
\end{equation*}
所以对于 $\delta$ 度正则图,根据引理 \ref{Lemma 1} 我们可以得到
\begin{equation*}
|\partial^\ast(S)| = \delta |S| - |E(S)|
\end{equation*}
即对于 $\forall k$,
$\min_{|S| = k} |\partial^\ast(S)| = \delta k - \max_{|S| = k} |E(S)|$。
因此,对于正则图,计算
\begin{equation*}
\min_{\substack{
	S \subseteq V \\
	|S| = k
}}|\partial^\ast(S)|
\end{equation*}
与导出边问题等价,从而与 EIP 等价。
\end{proof}

\begin{exercisewithanswer}
找出 $m = |E_{Q_d}|$ 对应的公式。
\end{exercisewithanswer}

\begin{answer}[解答]
将两个 $Q_{d - 1}$ 中的对应点一一相连即可构成一个 $Q_d$,因此有
\begin{equation*}
m = |E_{Q_d}| = \begin{cases}
	2 \left|E_{Q_{d - 1}}\right| + 2^{d - 1} & d > 0 \\
	0                                        & d = 0
\end{cases}
\end{equation*}
可以得到
\begin{equation*}
m = |E_{Q_d}| = d \cdot 2^{d - 1}, d \ge 0
\end{equation*}
\end{answer}

\begin{exercisewithanswer}
证明 $d$ 阶立方的任意 $c$ 阶子立方与 $c$ 阶立方同构。
\end{exercisewithanswer}

\begin{proof}[证明]
根据子立方的定义,$d$ 阶立方的一个 $c$ 阶子立方的顶点集有 $(d - c)$ 个固定坐标的值相同。
因此其中的两个顶点是否有边相连仅由剩下的 $c$ 个坐标决定,
即两个顶点对应一条边当且仅当剩下的 $c$ 个坐标中正好有一位不相同。
令 $Q_d^c$ 表示 $d$ 阶立方的一个 $c$ 阶子立方,
我们可以构造一个双射函数 $\phi \colon Q_d^c \rightarrow Q_c$,
对于 $\forall x \in V_{Q_d^c}$,
$\phi(x)$ 是将 $x$ 的 $(d - c)$ 个固定坐标移除后得到的 $0$ 和 $1$ 的 $c$ 元组。
显而易见,对于 $\forall x, y \in V_{Q_d^c}$,
$x$ 和 $y$ 对应一条边当且仅当 $\phi(x)$ 和 $\phi(y)$ 对应一条边,
即 $\phi$ 是 $Q_d^c$ 到 $Q_c$ 的一个图同构。
\end{proof}

\begin{exercisewithanswer}
计算 $d$ 阶立方有多少个 $c$ 阶子立方?
\end{exercisewithanswer}

\begin{answer}[解答]
首先我们从 $d$ 个坐标中选出 $(d - c)$ 个固定坐标,有 $\binom{d}{d - c}$ 种选法。
接下来我们为这 $(d - c)$ 个固定坐标确定一个值($0$ 或 $1$),构成一个 $c$ 阶子立方,
有 $2^{d - c}$ 种选法。
因此一个 $d$ 阶立方有 $\binom{d}{d - c} \cdot 2^{d - c}$ 个 $c$ 阶子立方。
\end{answer}

\begin{exercisewithanswer}
证明一个 $c$ 阶子立方的所有邻居都互不相交。
\end{exercisewithanswer}

\begin{proof}[证明]
令 $Q_d^c$ 表示 $d$ 阶立方的任意一个 $c$ 阶子立方,
$Q_d^{c, i}$ 表示 $Q_d^c$ 的一个邻居,
它的 $(d - c)$ 个固定坐标中第 $i$ 位与 $Q_d^c$ 不同($1 \le i \le d - c$)。
则 $Q_d^c$,
以及 $Q_d^c$ 的任意两个不同邻居 $Q_d^{c, a}$ 和 $Q_d^{c, b}$(不妨假设 $a < b$)
的顶点集可表示为
\begin{align*}
V_{Q_d^c}      & = \{x \in V_{Q_d} \mid
		   x_{p_1} = v_1, x_{p_2} = v_2, \dots, x_{p_{d - c}} = v_{d - c}\} \\
V_{Q_d^{c, a}} & = \{x \in V_{Q_d} \mid
		   \dots, x_{p_a} = 1 - v_a, \dots, x_{p_b} = v_b, \dots\} \\
V_{Q_d^{c, b}} & = \{x \in V_{Q_d} \mid
		   \dots, x_{p_a} = v_a, \dots, x_{p_b} = 1 - v_b, \dots\}
\end{align*}
其中对于 $\forall i \in \{1, 2, \dots, d - c\}$,
$v_i \in \{0, 1\}$,$p_i \in \{1, 2, \dots, d\}$ 且互不相同。
很显然,若 $\exists x$ 同时满足 $x \in V_{Q_d^{c, a}}$ 和 $x \in V_{Q_d^{c, b}}$,
那么 $x_{p_a}$ 和 $x_{p_b}$ 的取值将无法得到满足。
因此 $V_{Q_d^{c, a}} \cap V_{Q_d^{c, b}} = \emptyset$。
\end{proof}

\begin{exercisewithanswer}
证明一个 $c$ 阶子立方的两个不同邻居(的顶点)之间没有边相连。
\end{exercisewithanswer}

\begin{proof}[证明]
$Q_d^c$、$Q_d^{c, a}$ 和 $Q_d^{c, b}$ 的记号同上一证明。
由上一证明可知,
对于 $\forall x \in V_{Q_d^{c, a}}$ 和 $\forall y \in V_{Q_d^{c, b}}$,
\begin{equation*}
\begin{cases}
x_{p_a} \neq y_{p_a} \\
x_{p_b} \neq y_{p_b}
\end{cases}
\end{equation*}
因此 $x$ 和 $y$ 之间没有边相连。
\end{proof}

\begin{exercisewithanswer}
计算 $d$ 阶立方的一个 $c$ 阶子立方有多少个邻居?
\end{exercisewithanswer}

\begin{answer}[解答]
根据邻居的定义,$d$ 阶立方的一个 $c$ 阶子立方有 $(d - c)$ 个邻居。
\end{answer}

\begin{exercisewithanswer}
证明如果 $S \subseteq V_{Q_d}$ 是立方集,那么它的补集 $V_{Q_d} − S$ 也是立方集。
\end{exercisewithanswer}

\begin{proof}[证明]
假设 $V_{Q_d} − S$ 不是立方集,
即若把 $V_{Q_d} − S$ 视为互不相交的子立方的并集,
则至少存在一个 $c_i$ 阶子立方 $Q_d^{c_i}$ 和一个 $c_j$ 阶子立方 $Q_d^{c_j}$
满足 $c_i \le c_j$,且 $Q_d^{c_i}$ 不处在 $Q_d^{c_j}$ 的任意邻居中。
那么 $S$ 中必然存在一个 $c_j$ 阶子立方 ${Q_d^{c_j}}'$,
它是 $Q_d^{c_j}$ 的邻居,但不为 $Q_d^{c_i}$ 提供邻居。
如果我们把 $Q_d^{c_i}$ 归入 $S$,
把 ${Q_d^{c_j}}'$ 中与 $Q_d^{c_i}$ 对应的顶点归入 $V_{Q_d} − S$,
那么 $Q_d^{c_i}$ 将为 $S$ 提供更多的内部边,这便与 $S$ 是立方集相矛盾。
因此 $V_{Q_d} − S$ 也是立方集。
\end{proof}

\begin{exercisewithanswer}
证明任意两个 $k$-立方集都同构,
即存在一个 $d$ 阶立方的自同构能把一个($k$-立方集)映射到另一个。
\end{exercisewithanswer}

\begin{proof}[证明]
令 $S_1, S_2$ 表示 $d$ 阶立方的任意两个 $k$-立方集,$|S_1| = |S_2| = k$。
根据练习 \ref{Exercise 8},$V_{Q_d} - S_1$ 和 $V_{Q_d} - S_2$ 是 $(2^d - k)$-立方集。
我们把 $S_1, S_2$ 和 $V_{Q_d} - S_1, V_{Q_d} - S_2$ 都表示成互不相交的子立方的并集
\begin{align*}
S_1           & = \bigcup_{i = 1}^K V_{Q_d^{c_i, 1}} \\
S_2           & = \bigcup_{i = 1}^K V_{Q_d^{c_i, 2}} \\
V_{Q_d} - S_1 & = \bigcup_{i = 1}^{K'} V_{Q_d^{c_i, 3}} \\
V_{Q_d} - S_2 & = \bigcup_{i = 1}^{K'} V_{Q_d^{c_i, 4}}
\end{align*}
由于 $k$ 和 $2^d - k$ 的二进制表示形式都是唯一的,
我们可以很容易地知道 $S_1$ 中的子立方和 $S_2$ 中的子立方一一对应,
$V_{Q_d} - S_1$ 和 $V_{Q_d} - S_2$ 同理。

接下来我们构造一个映射 $\phi \colon Q_d \rightarrow Q_d$,满足
\begin{align*}
\forall i \in \{1, 2, \dots, K\}, \phi(Q_d^{c_i, 1}) = Q_d^{c_i, 2} \\
\forall i \in \{1, 2, \dots, K'\}, \phi(Q_d^{c_i, 3}) = Q_d^{c_i, 4}
\end{align*}
显而易见,$\phi$ 是一个双射函数,因此是一个 $Q_d$ 上的自同构。
\end{proof}

\begin{exercisewithanswer}
完成引理 \ref{Lemma 3} 情况 \ref{Lemma 3 Case 3} 的证明(右边的不等式)。
\end{exercisewithanswer}

\begin{proof}[证明]
同样分两种情况:
\begin{enumerate}[(1)]
	\item 当 $t < 2^{d - 1}$ 时,我们有
		\begin{align*}
		E(k + t) - E(k) & = \left[E(k + t) - E(2^{d - 1})\right] +
				    \left[E(2^{d - 1}) - E(k)\right] \\
				& = \left[E(k + t - 2^{d - 1}) + k + t - 2^{d - 1}\right] +
				    \left[E(2^{d - 1}) - E(k)\right] \\
				& = \left[E(k + t - 2^{d - 1}) - E(k)\right] +
				    E(2^{d - 1}) + k + t - 2^{d - 1} \\
				& < E(2^{d - 1}) - E(2^{d - 1} - t) + t + k - 2^{d - 1} \\
				& = E(2^d) - E(2^d - t) + k - 2^{d - 1} \\
				& < E(2^d) - E(2^d - t)
		\end{align*}
	\item 当 $t \ge 2^{d - 1}$ 时,我们有
		\begin{align*}
		E(k + t) - E(k) & = \left[E(k + t) - E(2^{d - 1})\right] +
				    \left[E(2^{d - 1}) - E(k)\right] \\
				& = \left[E(k + t - 2^{d - 1}) + k + t - 2^{d - 1}\right] +
				    \left[E(2^{d - 1}) - E(k)\right] \\
				& = \left[E(k + t - 2^{d - 1}) - E(k)\right] +
				    E(2^{d - 1}) + k + t - 2^{d - 1} \\
				& < E(2^{d - 1}) - E(2^d - t) + E(2^{d - 1}) +
				    k + t - 2^{d - 1} \\
				& = 2 E(2^{d - 1}) + k + t - 2^{d - 1} - E(2^d - t) \\
				& < 2 E(2^{d - 1}) + 2^{d - 1} - E(2^d - t) \\
				& = E(2^d) - E(2^d - t)
		\end{align*}
\end{enumerate}
\end{proof}

\begin{exercisewithanswer}
假设线形机箱实际上是一条实线,上面有 $n$ 个位置 $s_1 < s_2 < \cdots < s_n$ 用来放置元件。
布局 $\eta$ 将顶点 $v \in V$ 放置在位置 $s_{\eta(v)}$ 上,证明 $\eta$ 有总线长为
\begin{equation*}
wl(\eta) = \sum_{k = 0}^n (s_{k + 1} - s_k) |\Theta(S_k(\eta))|
\end{equation*}
\end{exercisewithanswer}

\begin{proof}[证明]
练习 \ref{Exercise 11} 的证明。
\end{proof}

\begin{exercisewithanswer}
证明 $wl(Q_d) = 2^{d - 1} (2^d - 1)$。
\end{exercisewithanswer}

\begin{proof}[证明]
练习 \ref{Exercise 12} 的证明。
\end{proof}

\begin{exercisewithanswer}
对于练习 \ref{Exercise 11} 中的更一般的线形机箱,
证明 $lex$ 同时也是它的 $Q_d$ 上线长问题的一个解。
\end{exercisewithanswer}

\begin{proof}[证明]
练习 \ref{Exercise 13} 的证明。
\end{proof}

\begin{exercisewithanswer}
找出 $\underline{DB_4}$ 上所有的 EIP 的解集,其中 $k \le 8$(无需证明)。
\end{exercisewithanswer}

\begin{answer}[解答]
练习 \ref{Exercise 14} 的解答。
\end{answer}

\begin{exercisewithanswer}
在阅读定理 \ref{Theorem 3} 的证明之前,先证明 $a = 1$ 时的特例。
\end{exercisewithanswer}

\begin{proof}[证明]
练习 \ref{Exercise 15} 的证明。
\end{proof}

\begin{exercisewithanswer}
给出引理 \ref{Lemma 6} 的证明。
\end{exercisewithanswer}

\begin{proof}[证明]
引理 \ref{Lemma 6} 的证明。
\end{proof}

%%%%%%%%%%%%%%%%%%%%%%%%%%%%%%%%%%%%%%%%%%%%%%%%%%%%%%%%%%%%%%%%%%%%%%%%%%%%%%%%
% document bibliography %%%%%%%%%%%%%%%%%%%%%%%%%%%%%%%%%%%%%%%%%%%%%%%%%%%%%%%%
%%%%%%%%%%%%%%%%%%%%%%%%%%%%%%%%%%%%%%%%%%%%%%%%%%%%%%%%%%%%%%%%%%%%%%%%%%%%%%%%
\begin{thebibliography}{9}
\label{Bibliography}
\addcontentsline{toc}{section}{\refname} % add bibliography to ToC

\bibitem{Kautz.1954}
	W. H. Kautz,
	\emph{Optimized data encoding for digital computers},
	Convention Record I. R. E., 1954, pp. 47-57

\bibitem{Bernstein.1967}
	A. J. Bernstein,
	\emph{Maximally connected arrays on the n-cube},
	SIAM J. Appl. Math., vol. 15, pp. 1485–1489, 1967

\bibitem{Steiglitz.1965}
	K. Steiglitz, A. J. Bernstein,
	\emph{Optimal binary coding of ordered numbers},
	J. SIAM, vol. 13, pp. 441-443, 1965

\bibitem{Golomb.1982}
	S. W. Golomb,
	\emph{Shift Register Sequences},
	Aegean Park Press, 1982

\bibitem{Harper.1964}
	L. H. Harper,
	\emph{Optimal assignments of numbers to vertices},
	J. SIAM, vol. 12, pp. 131-135, 1964

\bibitem{Bezrukov.1997}
	S. L. Bezrukov,
	\emph{On k-partitioning the n-cube},
	Proc. International Conference on Graph Theory Concepts in Computer Science,
	Como, Italy, 1997

\bibitem{Lehman.1963}
	A. Lehman,
	\emph{A result on rearrangements},
	Israel J. Math, vol. 1, pp. 22-28, 1963

\bibitem{Harper.1970}
	L. H. Harper,
	\emph{Chassis layout and isoperimetric problems},
	Jet Propulsion Lab. SPS, vol. 11, pp. 37-66, 1970

\bibitem{Polya.1954}
	G. Polya,
	\emph{Mathematics and Plausible Reasoning} (2 vols.),
	Princeton University Press, 1954

\end{thebibliography}

\end{document}
