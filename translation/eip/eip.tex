%%%%%%%%%%%%%%%%%%%%%%%%%%%%%%%%%%%%%%%%%%%%%%%%%%%%%%%%%%%%%%%%%%%%%%%%%%%%%%%%
% document preamble %%%%%%%%%%%%%%%%%%%%%%%%%%%%%%%%%%%%%%%%%%%%%%%%%%%%%%%%%%%%
%%%%%%%%%%%%%%%%%%%%%%%%%%%%%%%%%%%%%%%%%%%%%%%%%%%%%%%%%%%%%%%%%%%%%%%%%%%%%%%%
\documentclass[12pt, a4paper]{article}

\usepackage[UTF8]{ctex}			% for Chinese environment
\usepackage[svgpath=./images/]{svg}	% for svg images
\usepackage{enumerate}			% for customized list style
\usepackage{hyperref}			% for bookmarks
\usepackage{amsmath}			% for mathematical features
\usepackage{amssymb}			% for mathematical symbols
\usepackage{amsthm}			% for theorems


\author{L. H. Harper}
\date{2016 年 3 月 11 日}

% hyperref options
\hypersetup{
	hidelinks,			% hide links
	pdfstartview={FitH},		% fit width
	pdftitle={Chinese Translation	% PDF title
		of Chapter 1 of Global Methods for Combinatorial Isoperimetric
		Problems},
	pdfauthor={Zhuojia Shen}	% PDF althor
}

% theorems
\newtheorem{theorem}{定理}
\newtheorem{lemma}{引理}
\newtheorem{corollary}{推论}
\newtheorem{exercise}{练习}


%%%%%%%%%%%%%%%%%%%%%%%%%%%%%%%%%%%%%%%%%%%%%%%%%%%%%%%%%%%%%%%%%%%%%%%%%%%%%%%%
% document body %%%%%%%%%%%%%%%%%%%%%%%%%%%%%%%%%%%%%%%%%%%%%%%%%%%%%%%%%%%%%%%%
%%%%%%%%%%%%%%%%%%%%%%%%%%%%%%%%%%%%%%%%%%%%%%%%%%%%%%%%%%%%%%%%%%%%%%%%%%%%%%%%
\begin{document}

\title{边等周问题}
\maketitle

\section{基本定义}
\label{Section 1}

一个图(Graph)$G = (V, E, \partial)$ 由顶点集 $V$、边集 $E$ 和标识每一条边对应一对顶点
(可以相同)的边界函数 $\partial \colon E \rightarrow (_1^v) \cup (_2^v)$ 构成。
图经常用图表(Diagram)来表示,其中顶点对应图表中的点,边对应图表中连接一对点的曲线。
对任意的图 $G$ 和 $S \subseteq V$,我们定义:
$$\Theta(S) = \{e \in E \mid \partial(e) = \{v, w\}, v \in S, w \notin S\},$$
并称之为 $S$ 的边界(Edge-boundary)。那么对于一个给定的图 $G$ 和 $k \in \mathbb{Z}^+$,
边等周问题(Edge-Isoperimetric Problem, EIP)是对于所有的 $S \in V$ 并且 $|S| = k$,
找出最小的 $|\Theta(S)|$。注意 $|\Theta(S)|$ 是不变量(Invariant),即如果
$\phi \colon G \rightarrow H$ 表示一个图同构,那么对于 $\forall S \in V_G$,都有
$|\Theta(\phi(S))| = |\Theta(S)|$。因此,在一个自同构下等价的顶点子集有相同的边界。

自环,即只对应一个顶点的边,与 EIP 无关,因此我们将忽略它们。我们大多数的图(但并非全部)
是简单图(Ordinary Graph),即不含自环和多重边。一个简单图的表示可以缩短至 $(V, E)$,
其中 $E \in (_2^v)$,$\partial$ 则是隐含的。

\section{例子}
\label{Section 2}

\subsection{$K_n$,$n$-完全图}
\label{Subsection 2.1}

$K_n$ 有 $n$ 个顶点且 $E = (_2^v)$,即每一对不同顶点之间都有一条边。对于每一个 $S \in V$
且 $|S| = k$,$|\Theta(S)| = |S \times (V − S)| = k(n − k)$。因此 $K_n$ 的 EIP
很简单:任意一个 $k$-集都是解。

\subsection{$\mathbb{Z}_n$,$n$-环}
\label{Subsection 2.2}

对于 $\mathbb{Z}_n$,有 $V = \{0, 1, \dots, n − 1\}$ 和
$E = \{\{i, j\} \mid i − j \equiv  \pm 1(\models n)\}$。因此,$\mathbb{Z}_3 = K_3$,
$\mathbb{Z}_4$ 的图表见图 \ref{Figure 1}。

\begin{figure}
	\centering
	% \includesvg{figure-1}
	\caption{$\mathbb{Z}_4$ 的图表}
	\label{Figure 1}
\end{figure}

现在我们从下面的一般性说明(这些说明在后面也会有用),来推导 $\mathbb{Z}_4$ 以及
$\mathbb{Z}_n$ 上 EIP 的解:

\begin{enumerate}[(1)]
	\item 对于 $|S| = k = 0$,在任意的图里都只有一个子集,即空集 $\emptyset$。
		因此 $\min_{|S| = 0}|\Theta(S)| = |\Theta(\emptyset)| = 0$。
	\item 对于 $k = |V| = n$,也只有一个子集,即 $V$。
		因此 $\min_{|S| = n}|\Theta(S)| = |\Theta(V)| = 0$。
	\item 一个图被称为 $\delta$ 度正则图 (Regular of Degree $\delta$),
		当它的每个顶点都正好对应 $\delta$ 条边。在一个正则图中,
		如果 $|S| = k = 1$ 那么 $\Theta(S) = \delta$,
		因此任意单点集都是一个解集。$\mathbb{Z}_n$ 是 $2$ 度正则图;
		然而对于 $n = 4$ 且 $k = 2$ 的情况,有两个集合并不在 $\mathbb{Z}_n$
		的对称性下等价:$\{0, 1\}$ 和 $\{0, 2\}$。
		所有其他的 $2$-集都与这两者中的一个等价。$|\Theta(\{0, 1\})| = 2$,
		$|\Theta(\{0, 2\})| = 4$,因此 $\min_{|S| = 2}|\Theta(S)| = 2$。
	\item 对于 $\forall G$ 和 $\forall S \subseteq V$,有
		$$\Theta(V − S) = \Theta(S),$$
		因此对于 $k > \frac{1}{2}|V|$,
		$\min_{|S| = k}|\Theta(S)| = \min_{|S| = n − k}|\Theta(S)|$,
		其中 $n = |V|$。这样我们就完成了 $\mathbb{Z}_4$ 上 EIP 的解,
		见下表总结。
		\begin{center}
			\begin{tabular}{ c | c c c c c }
			$k$                         & 0 & 1 & 2 & 3 & 4 \\
			\hline
			$\min_{|S| = k}|\Theta(S)|$ & 0 & 2 & 2 & 2 & 0 \\
			\end{tabular}
		\end{center}
	\item 令
		$$E(S) = \{e \in E \mid \partial(e) = \{v, w\}, v \in S, w \in S\}。$$
		$E(S)$ 被称为 $S$ 的导出边(Induced Edges)。
		对于一个图,导出边问题(Induced Edge Problem)是对于所有的
		$S \subseteq V$ 并且 $|S| = k$,找出最大的 $|E(S)|$。
\end{enumerate}

\begin{lemma}
\label{Lemma 1}
如果图 $G = (V, E, \partial)$ 是一个 $\delta$ 度正则图,
那么对于 $\forall S \subseteq V$,都有
$$|\Theta(S)| + 2|E(S)| = \delta|S|。$$
\end{lemma}

\begin{proof}
$\delta|S|$ 表示 $S$ 对应的边数,然而出现在 $E(S)$ 中的边被计算了两次。
\end{proof}

\begin{corollary}
\label{Corollary 1}
如果 $G$ 是一个正则图,
那么 $S \subseteq V$ 是导出边问题的一个解当且仅当它同时也是 EIP 的解。
而且,对于 $\forall k$,$\min_{|S| = k}|\Theta(S)| = \delta k − 2\max_{|S| = k}|E(S)|$。
\end{corollary}

那么对于正则图,EIP 和导出边问题是等价的,我们将视它们为可等价互换的问题。
通常 EIP 出现在实际应用中,而导出边问题则更易证明。EIP 还存在第三种自然变体:
对于 $S \subseteq V$,令
$$\partial^\star(S) = \{e \in E \mid \partial(e) \cap S \neq \emptyset\},$$
即 $S$ 中的点对应的边的集合。

\begin{exercise}
\label{Exercise 1}
对于正则图,证明计算
$$\min_{\substack{
	S \subseteq V \\
	|S| = k \\
}}|\partial^\star(S)|$$
与 EIP 等价。
\end{exercise}

回想一下,一棵树(Tree)是连通无环图。一个无环图也被称为森林(Forest),因为它是
树——其连通分支的并集。

\begin{lemma}
\label{Lemma 2}
有 $n$ 个顶点的树的边数为 $n − 1$。有 $n$ 个顶点的森林的导出边数则是 $n - t$,
其中 $t$ 为其连通分支数。
\end{lemma}

$\mathbb{Z}_n$ 的任意适当的子集 $S$ 都可以导出一个无环图,
因此 $max_{|S| = k|}\Theta(S)|$ 会出现于当 $S$ 是一个连通的集合,即区间时。
因而,如果 $0 < k < n$,$\min_{|S| = k}|\Theta(S)| = 2k − 2(k − 1) = 2$。

\subsection{$Q_d$,$d$ 阶立方}
\label{Subsection 2.3}

$d$ 阶立方 $Q_d$ 有顶点集 $\{0, 1\}^d$,即 $\{0, 1\}$ 的 $d$ 倍笛卡尔积。
因此 $n = |V_{Q_d}| = 2^d$。$Q_d$ 中的两个顶点($0$ 和 $1$ 构成的 $d$ 元组)对应一条边,
当且仅当它们正好有一位不相同。

\begin{exercise}
\label{Exercise 2}
找出 $m = |E_{Q_d}|$ 对应的公式。
\end{exercise}

$Q_1$ 和 $K_2$ 同构,$Q_2$ 和 $\mathbb{Z}_4$ 同构,这些都已经可以通过 EIP 得到解决。
一个 $3$ 阶立方有 $8$ 个顶点,$12$ 条边,以及 $6$ 个平面。
$Q_3$ 的图表事实上是一个 $3$ 阶立方的投影,见图 \ref{Figure 2}。

\begin{figure}
	\centering
	% \includesvg{figure-2}
	\caption{$Q_3$ 的图表}
	\label{Figure 2}
\end{figure}

我们可以用在前两个例子中发展出来的简便工具来解决 $Q_3$ 上的 EIP。
首先观察到 $Q_3$ 有围长(最短回路的长度)为 $4$:
因为 $3$ 阶立方的对称群是传递的,所以任意顶点都和其他顶点一样。
从一个顶点出发勾勒出路径,我们可以看到不存在长度为 $3$ 的闭合回路。因此对于
$1 \le k \le 3$,根据引理 \ref{Lemma 1} 和引理 \ref{Lemma 2},我们有
$$\begin{array}{ r l }
\min\limits_{|S| = k}|\Theta(S)| & = 3k - 2\max\limits_{|S| = k}|E(S)| \\
                                 & = 3k - 2(k - 1) = k + 2
\end{array}。$$
对于 $k = 4$,要么 $S$ 导出一个回路,在这种情况下是一个 $4$-环,并且 $|\Theta(S)| = 4$;
要么 $S$ 导出一个无环图,并且根据上述可以得到 $|\Theta(S)| \ge 6$。
对于 $k > 4 = \frac{8}{2}$,我们根据下面这个式子得到解
$$\min\limits_{|S| = k}|\Theta(S)| = \min\limits_{|S| = 8 - k}|\Theta(S)|。$$
最后的解见下表总结。
\begin{center}
	\begin{tabular}{ c | c c c c c c c c c }
	$k$                         & 0 & 1 & 2 & 3 & 4 & 5 & 6 & 7 & 8 \\
	\hline
	$\min_{|S| = k}|\Theta(S)|$ & 0 & 3 & 4 & 5 & 4 & 5 & 4 & 3 & 0 \\
	\end{tabular}
\end{center}

为了将这个 EIP 的解推广到 $Q_d$($d > 3$),我们需要利用一些关于立方的简单结论,
这些结论的证明将被留作练习。
一个 $d$ 阶立方的 $c$ 阶子立方($c$-subcube of the $d$-cube)是 $Q_d$ 的子图,
导出自一个包含所有在 $d − c$ 个坐标下有相同(固定)值的顶点的集合。

\begin{exercise}
\label{Exercise 3}
证明 $d$ 阶立方的任意 $c$ 阶子立方与 $c$ 阶立方同构。
\end{exercise}

\begin{exercise}
\label{Exercise 4}
计算 $d$ 阶立方有多少个 $c$ 阶子立方?
\end{exercise}

一个 $d$ 阶立方的 $c$ 阶子立方的邻居(Neighbor)是在 $d − c$
个固定坐标下有正好一个坐标不同的任意 $c$ 阶子立方。

\begin{exercise}
\label{Exercise 5}
证明一个 $c$ 阶子立方的所有邻居都互不相交。
\end{exercise}

\begin{exercise}
\label{Exercise 6}
证明一个 $c$ 阶子立方的两个不同邻居(的顶点)之间没有边相连。
\end{exercise}

\begin{exercise}
\label{Exercise 7}
计算一个 $d$ 阶立方的 $c$ 阶子立方有多少个邻居?
\end{exercise}

$Q_d$ 上的 EIP 最初由数据传输上的问题推动。
W. H. Kautz \cite{Kautz}、E. C. Posner 和本书作者的研究引发了一种猜测:
字典序编号方式(Lexicographic Numbering)的起始部分,即
$$lex(x) = 1 + \sum_{i = 1}^dx_i2^{i - 1}, x \in V_{Q_d}$$
是解集,然而这要如何证明?一个显而易见的尝试方法是对维度 $d$ 使用归纳法。
数学归纳法有一个看似自相矛盾的属性,它往往更容易证明一个更强的定理,
因为一旦最初的情况得到验证,人们就可以假设该定理对归纳参数的较低值为真,
以便建立下一次归纳。因此一个较强的假设可以产生一个更简单的证明。
在本例中,该策略引发了这种推测——以下归纳步骤会生成 \emph{所有} 解集:

\begin{enumerate}[(1)]
	\item 首先从空集 $\emptyset$ 开始。
	\item 对于已经构建好的集合 $S \subset V_{Q_d}$,
		选择任意一个 $x \in V_{Q_d} − S$ 来扩充 $S$,使得导出边数的增量最大,即
		$$|E(S \cup \{x\})| − |E(S)|。$$
\end{enumerate}

添加任意 $x \in V_{Q_d}$ 都可使 $\emptyset$ 的增量最大,
因为 $|E(\{x\})| − |E(\emptyset)| = 0$。$\{x\}$ 的增量则必须是 $x$ 的相邻顶点。
那么对于 $k > 2$ 的 $k$-集,情况又该如何呢?答案是如果 $k = 2^c$,
那么该集合必然是 $c$ 阶子立方。我们刚刚已经验证了 $c = 0$ 和 $1$ 的情况。
假设这对 $0, 1, \dots, c − 1$ 都成立,为了扩充一个 $2^{c−1}$-集,
即一个 $(c − 1)$ 阶子立方,我们只能选择一个使 $|E(S)|$ 增量为 $1 $的顶点,
即一个相邻的 $(c − 1)$ 阶子立方中的任意顶点。从一个相邻的子立方中选出一个顶点后,
我们必须继续从同一个子立方中选顶点,直到用尽它所有的顶点,
这是因为所选择的子立方中总会存在一个顶点使得 $|E(S \cup {x})| − |E(S)| \ge 2$,
而其他子立方中的任意顶点都有 $|E(S \cup {x})| − |E(S)| \le 2$。
当我们用尽所有相邻的 $(c − 1)$ 阶子立方时,我们就得到了一个 $c$ 阶子立方。

\begin{thebibliography}{99}

\bibitem{Kautz}
	W. H. Kautz,
	\emph{Optimized data encoding for digital computers},
	Convention Record I. R. E., 1954, pp. 47-57

\bibitem{Bernstein}
	A. J. Bernstein,
	\emph{Maximally connected arrays on the n-cube},
	SIAM J. Appl. Math., vol. 15, no. 6, pp. 1485–1489, 1967

\bibitem{Steiglitz}
	K. Steiglitz, A. J. Bernstein,
	\emph{Optimal binary coding of ordered numbers},
	J. SIAM, vol. 13, no. 2, pp. 441-443, 1965

\bibitem{Golomb}
	S. W. Golomb,
	\emph{Shift Register Sequences},
	Aegean Park Press, 1982

\bibitem{Harper}
	L. H. Harper,
	\emph{Optimal assignments of numbers to vertices},
	J. SIAM, vol. 12, no. 1, pp. 131-135, 1964

\bibitem{Bezrukov}
	S. L. Bezrukov,
	\emph{On k-partitioning the n-cube},
	Proc. International Conference on Graph Theory Concepts in Computer Science, Como, Italy, 1997

\end{thebibliography}

\end{document}
